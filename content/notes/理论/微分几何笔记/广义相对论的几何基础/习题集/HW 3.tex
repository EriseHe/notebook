\documentclass[12pt]{article}

% Adjusted page margins to reduce vertical space
\usepackage[top=0.6in, bottom=1in,left=1in, right=1.2in]{geometry}

% Formatting and layout packages
\usepackage{titlesec}
\usepackage{lipsum}
\usepackage{mathpazo} % Font package for aesthetics

\usepackage{mdframed}

\newmdenv[
  topline=true, 
  bottomline=true,
  rightline=true, 
  leftline=true, 
  linecolor=black, 
  linewidth=0.8pt,
  backgroundcolor=white,
  innertopmargin=8pt,
  innerbottommargin=8pt,
  skipabove=8pt,
  skipbelow=8pt,
  splitbottomskip=2pt,  % Ensures proper spacing when the box splits across pages
  splittopskip=2pt,
]{solutionbox}

% Auto-wrap any "\item[Solution.]" in a solutionbox until next item at same level
\usepackage{xparse}
\usepackage{xstring}
\usepackage{etoolbox}

% Track nesting depth across list environments
\newcounter{solution@depth}
\newcounter{solution@level}
\newif\ifinsolutionbox
\insolutionboxfalse

% Increase/decrease depth for enumerate/itemize/description
\AtBeginEnvironment{enumerate}{\stepcounter{solution@depth}}
\AtEndEnvironment{enumerate}{%
  % If a solutionbox was opened at this level, close it before exiting
  \ifinsolutionbox
    \ifnum\value{solution@depth}=\value{solution@level}\relax
      \end{solutionbox}%
      \insolutionboxfalse
    \fi
  \fi
  \addtocounter{solution@depth}{-1}%
}
\AtBeginEnvironment{itemize}{\stepcounter{solution@depth}}
\AtEndEnvironment{itemize}{\addtocounter{solution@depth}{-1}}
\AtBeginEnvironment{description}{\stepcounter{solution@depth}}
\AtEndEnvironment{description}{\addtocounter{solution@depth}{-1}}

% Intercept \item to open/close solution boxes based on label
\let\olditem\item
\RenewDocumentCommand{\item}{o}{%
  % If we're starting a new item at the same level as an open solutionbox, close it
  \ifinsolutionbox
    \ifnum\value{solution@depth}=\value{solution@level}\relax
      \end{solutionbox}%
      \insolutionboxfalse
    \fi
  \fi
  % If no optional label, just pass through
  \IfNoValueTF{#1}{%
    \olditem
  }{%
    % If the label is exactly "Solution.", open a new solutionbox
    \IfStrEq{#1}{Solution.}{%
      \olditem[]%
      \begin{solutionbox}%
      \insolutionboxtrue
      \setcounter{solution@level}{\value{solution@depth}}%
    }{%
      \olditem[#1]% other labels unchanged
    }%
  }%
}



% Packages for figures and tables
\usepackage{graphicx}
\usepackage{subcaption}
\usepackage{svg}
\usepackage{booktabs}
\usepackage{float}

% Math-related packages
\usepackage{amsmath, amssymb, amsthm}

% TikZ for diagrams
\usepackage{tikz}
\usetikzlibrary{arrows,arrows.meta,angles,quotes,calc}

% Declare supported graphic file types
\DeclareGraphicsExtensions{.pdf,.jpg,.tif,.png}

% Theorem and definition environments
\newtheorem{theorem}{Theorem}
\newtheorem{corollary}[theorem]{Corollary}
\newtheorem{lemma}[theorem]{Lemma}
\newtheorem{definition}{Definition}

% Reduce unnecessary vertical spacing
\setlength{\parskip}{0pt}  % No extra space between paragraphs
\setlength{\topsep}{2pt}   % Reduce spacing before lists/theorems
\setlength{\partopsep}{0pt}
\raggedbottom  % Prevents vertical stretching
\pagenumbering{gobble}

% Load hyperref last to apply hidelinks properly
\usepackage[hidelinks]{hyperref}

%%%%%%%%%%%%%%%%%%%%%%%%%%%%%%%%
\begin{document}
\thispagestyle{plain}
\vspace{-4ex}  % Less aggressive spacing reduction


\begin{center}
\begin{tabular}{*{3}{c}}
    \parbox[t]{0.3\linewidth}{\centering\textbf{Problem Set 3}}
    & \parbox[t]{0.3\linewidth}{\centering\textbf{MATH 545\\General Relativity}}
    & \parbox[t]{0.3\linewidth}{\centering\textbf{Erise He}}\\[2em]
    \hline
\end{tabular}
\end{center}

\bigskip

%%%%%%%%%%%%%%%%%%%%%%%%%%%%%%%%%%%%%%%%%%%%%%%%%%%%%%%%%%%%%%%%%%%%%%%%%%%%%%
%%%%%%%%%%%%%%%%%%%%%%%%%%%%%%%%%%%%%%%%%%%%%%%%%%%%%%%%%%%%%%%%%%%%%%%%%%%%%%
\begin{enumerate}
  \item[Problem 1] \textbf{Geodesics in polar coordinates: Book problem P8.3}
  
  Our goal in this problem is to show that solutions to the geodesic equation in 2D polar coordinates on a flat plane are indeed straight lines.
  \item[(a)] We can specify an arbitrary line on a flat 2D surface by specifying the distance $b$ between the origin $O$ and the point $B$ on the line that is closest, and the angle $\alpha$ that the line $O B$ makes with the $x$ axis (see figure 8.5). Let us also define the pathlength $s$ to be zero at $B$ and increase as we go along the line in the direction counterclockwise around the origin. Using figure 8.5, argue that such a line can be described in polar coordinates as a function of $s$ by
  $$
  r^2=s^2+b^2
  $$
  and
  $$
  \theta=\alpha+\tan ^{-1}\left(\frac{s}{b}\right)
  $$

  \item[Solution.]
  We can direclty infer these equations from the figure, looking at the triangle $O B A$. Since it is a right triangle, we can use the Pythagorean theorem to have
  $$
  \begin{aligned} \quad O P & =r, \quad O B=b, \quad B P=s \\ \Rightarrow \quad r^2 & =s^2+b^2\end{aligned}
  $$
  and 
  $$
  \angle(O B, O P)=\tan ^{-1}\left(\frac{s}{b}\right)=\theta-\alpha
  $$
  so we have
  $$
  \theta = \alpha + \tan^{-1}\left(\frac{s}{b}\right)
  $$


  \item[(b)] Show that the geodesic equations for the polar coordinate system imply

  $$
  \frac{d \theta}{d s}=\frac{c}{r^2}, \quad \frac{d^2 r}{d s^2}=\frac{c^2}{r^3}
  $$

  where $c$ is some constant of integration.

  \item[Solution.]
  Given that $ds^2 = dr^2 + r^2 d\theta^2$, we have Lagrange's equation:

  $$\mathcal{L}=\frac{1}{2}\left(r^{\prime 2}+r^2 \theta^{\prime 2}\right)$$
  where $'=\frac{d}{d s}$. We compute EL equations. First for $\theta$:
  $$
  \begin{aligned}
  \frac{d}{d s}\left(\frac{\partial \mathcal{L}}{\partial \theta^{\prime}}\right)-\frac{\partial \mathcal{L}}{\partial \theta}&=0\\
  \frac{d}{d s}\left(r^2 \theta^{\prime}\right)-0&=0\\
  r^2 \theta^{\prime}&=c\\
  \theta^{\prime}&=\frac{c}{r^2}
  \end{aligned}
  $$
  and for $r$:
  $$
  \begin{aligned}
  \frac{d}{d s}\left(\frac{\partial \mathcal{L}}{\partial r^{\prime}}\right)-\frac{\partial \mathcal{L}}{\partial r}&=0\\
  \frac{d}{d s}\left(r^{\prime}\right)-\frac{\partial}{\partial r}\left(\frac{1}{2} r^2 \theta^{\prime 2}\right)&=0\\
  r^{\prime \prime}-r \theta^{\prime 2}&=0\\
  \frac{d^2 r}{d s^2}=\frac{c^2}{r^3}
  \end{aligned}
  $$
  where $\theta^{\prime}=c / r^2$ as we previously found.


  \item[(c)] In this case we cannot easily integrate equation 8.63 directly, but we can use the definition of pathlength (equation 8.16) to determine $d r / d s$. Show in this case that equations 8.16 and 8.62 together imply that

  $$
  \frac{d r}{d s}= \pm \sqrt{1-\frac{c^2}{r^2}}
  $$

  \item[Solution.]
  Equation 8.16 gives us the unit-speed condition:
  $$g_{\mu \nu} \frac{d x^\mu}{d s} \frac{d x^\nu}{d s}=+\left(\frac{d s}{d s}\right)^2=+1$$
  for $x^\mu=(r, \theta), g_{r r}=1, g_{\theta \theta}=r^2$, we have


  $$
  \begin{aligned}
   (r')^2+r^2(\theta')^2&=1 \\
  (r')^2+r^2\left(\frac{c}{r^2}\right)^2&=1 \\
  (r')^2&=1-\frac{c^2}{r^2} \\
  r'&= \pm \sqrt{1-\frac{c^2}{r^2}}
  \end{aligned}
  $$
  which is what we wanted to show.


  \item[(d)] This equation is essentially an integral of equation 8.63. Prove this by showing that taking the $s$-derivative of equation 8.64 yields equation 8.63.
  \item[Solution.] 
  
  $$
  \begin{aligned}
  r^{\prime \prime} & =\frac{d}{d s}\left[ \pm\left(1-\frac{c^2}{r^2}\right)^{\frac{1}{2}}\right] \\ & = \pm \frac{1}{2}\left(1-\frac{c^2}{r^2}\right)^{-\frac{1}{2}} \frac{d}{d s}\left(1-\frac{c^2}{r^2}\right) \\ & = \pm \frac{1}{2}\left(1-\frac{c^2}{r^2}\right)^{-\frac{1}{2}}\left(\frac{2 c^2}{r^3} r^{\prime}\right) \\ & = \pm \frac{c^2}{r^3}\left(1-\frac{c^2}{r^2}\right)^{-\frac{1}{2}} r^{\prime} \\ & = \pm \frac{c^2}{r^3}\left(1-\frac{c^2}{r^2}\right)^{-\frac{1}{2}}\left( \pm (1-\frac{c^2}{r^2})^{\frac{1}{2}}\right) \\ & =\frac{c^2}{r^3} \end{aligned}
  $$
  using $r^{\prime}= \pm \sqrt{1-\frac{c^2}{r^2}}$.

  \item[(e)] Note that equation 8.64 makes sense only if $r \geq c$. Show that if you isolate the $r$-dependent terms on one side and the $d s$ on the other side and integrate, you get $r^2=s^2+c^2$, which is the same as our original equation 8.60 if we identify the constant of integration $c$ as being the distance $b$ in figure 8.5.
  
  \item[Solution.]
  We can rewrite r' in (c) as:
  $$
  \frac{d r}{d s}=\frac{\sqrt{r^2-c^2}}{r}
  $$
  where $s=d s=\frac{r d r}{\sqrt{r^2-c^2}}$. Integrating, we get
  $$\int d s=\int \frac{r d r}{\sqrt{r^2-c^2}} \stackrel{u=r^2-c^2}{=} \int \frac{d u}{2 \sqrt{u}}=\sqrt{u}=\sqrt{r^2-c^2}+ C$$
  so we have $s=\sqrt{r^2-c^2}+ C$. Picking C to be $s=0$ when $r=c$, which means:
  $$s=\sqrt{r^2-c^2} \Longleftrightarrow r^2=s^2+c^2$$
  which is the same as equation 8.60.


  \item[(f)] Finally, show that if you substitute $r^2=s^2+c^2$ into the geodesic equation 8.62 , set $c=b$, and integrate, you get $\theta=\alpha+\tan ^{-1}(s / b)$ (where $\alpha$ is another constant of integration), which is our original equation 8.61. We have therefore shown that solutions to the geodesic equation for polar coordinates are simply straight lines of arbitrary orientation expressed in polar coordinates (as we might expect).

  \item[Solution.]
  
  Given $r^2=s^2+c^2$ and $c=b$ for this problem,

  $$
  \theta'=\frac{c}{r^2}=\frac{c}{s^2+b^2}
  $$


  Integrate to have

  $$
  \theta(s)=\int \frac{c d s}{s^2+b^2}=\tan ^{-1}\left(\frac{s}{b}\right)+C
  $$
  Then let $C = \alpha$, we have
  $$
  \theta(s)=\alpha+\tan ^{-1}\left(\frac{s}{b}\right)
  $$
  as desired.

\end{enumerate}
%%%%%%%%%%%%%%%%%%%%%%%%%%%%%%%%%%%%%%%%%%%%%%%%%%%%%%%%%%%%%%%%%%%%%%%%%%%%%%
\newpage%%%%%%%%%%%%%%%%%%%%%%%%%%%%%%%%%%%%%%%%%%%%%%%%%%%%%%%%%%%%%%%%%%%%%%
%%%%%%%%%%%%%%%%%%%%%%%%%%%%%%%%%%%%%%%%%%%%%%%%%%%%%%%%%%%%%%%%%%%%%%%%%%%%%%
\begin{enumerate}
  \item[Problem 2] \textbf{Geodesics II: Book problem P8.4} 
  \item[(a)] Show that the geodesic equation implies that

  $$
  0=g_{\alpha \beta} \frac{d u^\beta}{d \tau}+\left(\partial_\sigma g_{\alpha \beta}\right) u^\sigma u^\beta-\frac{1}{2}\left(\partial_\alpha g_{\mu \nu}\right) u^\mu u^\nu
  $$
  
  where $u^\alpha \equiv d x^\alpha / d \tau$ is the four-velocity of the object following the geodesic. (Hint: Remember that the metric depends implicitly on $\tau$. Why?)

  \item[Solution.]
  We consider the Lagrangian for a geodesic parameterized by $\tau$:
\[
L(x(\tau),\dot x(\tau))=\frac12\,g_{\mu\nu}(x)\,\dot x^\mu \dot x^\nu,
\qquad \dot x^\mu \equiv \frac{dx^\mu}{d\tau} \equiv u^\mu
\]
Euler–Lagrange for the coordinate $x^\alpha$ reads
\[
\frac{d}{d\tau}\!\left(\frac{\partial L}{\partial \dot x^\alpha}\right)
-\frac{\partial L}{\partial x^\alpha}=0.
\]
we compute the two terms, using $u^\mu=\dot x^\mu$ to have the momentum term and the position derivative term. First, for the \textbf{momentum term}:
\[
\frac{\partial L}{\partial \dot x^\alpha}
=\frac12\,g_{\mu\nu}\,\big(\delta^\mu_{\ \alpha}\dot x^\nu+\dot x^\mu \delta^\nu_{\ \alpha}\big)
=g_{\alpha\beta}\,\dot x^\beta
=g_{\alpha\beta}\,u^\beta
\]

and for the \textbf{position term}(here the metric depends on $x$, and thats why it is implicitly on $\tau$ along the path):
\[
\frac{\partial L}{\partial x^\alpha}
=\frac12\,(\partial_\alpha g_{\mu\nu})\,\dot x^\mu \dot x^\nu
=\frac12\,(\partial_\alpha g_{\mu\nu})\,u^\mu u^\nu
\]

we then take the total $\tau$–derivative in the first term and expand by the product rule:
\[
\frac{d}{d\tau}\!\left(g_{\alpha\beta}u^\beta\right)
=(\partial_\sigma g_{\alpha\beta})\,\frac{dx^\sigma}{d\tau}\,u^\beta
+ g_{\alpha\beta}\,\frac{d u^\beta}{d\tau}
=(\partial_\sigma g_{\alpha\beta})\,u^\sigma u^\beta
+ g_{\alpha\beta}\,\frac{d u^\beta}{d\tau}
\]

finally we insert these into EL:
\[
(\partial_\sigma g_{\alpha\beta})\,u^\sigma u^\beta
+ g_{\alpha\beta}\,\frac{d u^\beta}{d\tau}
- \frac12\,(\partial_\alpha g_{\mu\nu})\,u^\mu u^\nu
=0
\]

which is exactly what we wanted to show:
\[
\boxed{\;
0=g_{\alpha \beta}\frac{d u^\beta}{d \tau}
+\left(\partial_\sigma g_{\alpha \beta}\right) u^\sigma u^\beta
-\frac{1}{2}\left(\partial_\alpha g_{\mu \nu}\right) u^\mu u^\nu
\; }
\]

  \item[(b)] Use this result to prove that the geodesic equation implies that $\boldsymbol{u} \cdot \boldsymbol{u}$ is constant for any object following a geodesic (as must be, since $\boldsymbol{u} \cdot \boldsymbol{u}=-1$ by definition of $\boldsymbol{u})$. [Hint: Write out $d(\boldsymbol{u} \cdot \boldsymbol{u}) / d \tau$, multiply equation 8.65 by $u^\alpha$ and sum over $\alpha$, and compare.]
  
  \item[Solution.]
  
  Let $u\!\cdot\!u \equiv g_{\mu\nu}u^\mu u^\nu$, and we differentiate with respect to $\tau$:
\[
\frac{d}{d\tau}(u\!\cdot\!u)
= \frac{d}{d\tau}\big(g_{\mu\nu}u^\mu u^\nu\big)
= (\partial_\sigma g_{\mu\nu})\,u^\sigma u^\mu u^\nu
  + 2\,g_{\mu\nu}u^\mu\frac{d u^\nu}{d\tau}
\]
where the first term uses the chain rule $dg_{\mu\nu}/d\tau=(\partial_\sigma g_{\mu\nu})\,dx^\sigma/d\tau$. Now we take the result from (a),
\[
g_{\alpha\beta}\frac{d u^\beta}{d\tau}
= -(\partial_\sigma g_{\alpha\beta})\,u^\sigma u^\beta
  + \tfrac12(\partial_\alpha g_{\mu\nu})\,u^\mu u^\nu
\]
multiply by $2u^\alpha$, and sum over $\alpha$:
\[
2\,g_{\alpha\beta}u^\alpha\frac{d u^\beta}{d\tau}
= -2\,(\partial_\sigma g_{\alpha\beta})\,u^\alpha u^\sigma u^\beta
  + (\partial_\alpha g_{\mu\nu})\,u^\alpha u^\mu u^\nu
\]
Relabel dummy indices to compare with $\frac{d}{d\tau}(u\!\cdot\!u)$; noting
\(
(\partial_\sigma g_{\alpha\beta})\,u^\sigma u^\alpha u^\beta
= (\partial_\alpha g_{\mu\nu})\,u^\alpha u^\mu u^\nu
\)
by dummy-index renaming, we obtain
\[
\frac{d}{d\tau}(u\!\cdot\!u)
= (\partial_\alpha g_{\mu\nu})\,u^\alpha u^\mu u^\nu
  + \Big[-2(\partial_\alpha g_{\mu\nu})\,u^\alpha u^\mu u^\nu
         + (\partial_\alpha g_{\mu\nu})\,u^\alpha u^\mu u^\nu\Big]
= 0
\]
Hence
\[
\boxed{\;\frac{d}{d\tau}(u\!\cdot\!u)=0\;}
\]
so $u\!\cdot\!u$ is constant along any geodesic.

\end{enumerate}
%%%%%%%%%%%%%%%%%%%%%%%%%%%%%%%%%%%%%%%%%%%%%%%%%%%%%%%%%%%%%%%%%%%%%%%%%%%%%%
\newpage%%%%%%%%%%%%%%%%%%%%%%%%%%%%%%%%%%%%%%%%%%%%%%%%%%%%%%%%%%%%%%%%%%%%%%
%%%%%%%%%%%%%%%%%%%%%%%%%%%%%%%%%%%%%%%%%%%%%%%%%%%%%%%%%%%%%%%%%%%%%%%%%%%%%%
\begin{enumerate}
  \item[Problem 3] \textbf{Non-flat metric: Book problem P9.7} Consider the following metric equation:
  
  $$
  d s^2=-d t^2+d r^2+R^2 \sinh ^2(r / R)\left(d \theta^2+\sin ^2 \theta d \phi^2\right)
  $$
  
  where $R$ is a constant with units of length.
  \item[(a)] What kinds of clocks and in what positions register the coordinate time $t$ ?
  
  \item[Solution.]
  If we set $dr=d\theta=d\phi=0$, and look at only the the clock at rest, then it is just:
  \[
    ds^2 = -dt^2 \quad\Longrightarrow\quad d\tau^2 \equiv -ds^2 = dt^2
  \]
  meaning that such a clock measures proper time $d\tau=dt$, so the coordinate time $t$ equals the proper time of any ideal clock at rest at any position $(r,\theta,\phi)$.

  
  \item[(b)] Does this metric describe a spherically symmetric spacetime? Carefully justify your response.

  \item[Solution.]
  
  Yes, such spacetime is spherically symmetric because we can re-write the metric as
  \[
  \begin{aligned}
    ds^2 = -dt^2 + dr^2 + a^2\, d\Omega^2 
  \end{aligned}
  \]
  where $a\equiv R\sinh\!\big(r/R\big)$, and $d\Omega^2 = d\theta^2+\sin^2\!\theta\,d\phi^2$.
  The angular part is basically the metric on a sphere multiplied by a function $a(r)^2$, which only depends on $r$. The coefficients are independent of $(\theta,\phi)$. 

  \item[(c)] Is the $r$ coordinate a radial coordinate or a circumferential coordinate? Is the circumference of a circle bigger than, equal to, or less than $2 \pi r$ ? Explain.
  
  \item[Solution.]
  The $r$–coordinate is a proper radial distance along a purely radial line ($dt=d\theta=d\phi=0$),
  \[
    ds^2 = dr^2 \;\;\Longrightarrow\;\; \int dr = r + \text{C}
  \]
  which is not the circumferential (area) radius, because the areal/circumferential radius is
  \[
    \rho(r) \equiv a(r) = R\sinh\!\big(r/R\big),
  \]
  since the metric on the $\mathbb{S}^2$ of constant $t,r$ is $\rho(r)^2 d\Omega^2$. The circumference of an equatorial circle ($\theta=\pi/2$) is
  \[
    C(r) = \int_0^{2\pi}\!\sqrt{g_{\phi\phi}}\,d\phi
          = \int_0^{2\pi}\!\rho(r)\,d\phi
          = 2\pi\,R\sinh\!\big(r/R\big).
  \]
  compare with $2\pi r$, since for $x>0$, we can show that $\sinh x > x$, hence
  \[
    C(r)=2\pi R\sinh\!\big(r/R\big) \;>\; 2\pi r \quad\text{for } r>0
  \]
  It is with equality only at $r=0$. Therefore circles have circumference larger than $2\pi r$.

  
  \item[(d)] If we were to write this metric tensor as a matrix, it would have no off-diagonal terms. What does this imply about the coordinate system?
  
  \item[Solution.]
  
  A diagonal metric (no off-diagonal terms) means the coordinate basis vectors are mutually orthogonal, or the the coordinates are an orthogonal curvilinear system. In particular, this means that $g_{ti}=0$ ($i=r,\theta,\phi$) implies that the time coordinate lines are orthogonal to the spatial slicing. 

\end{enumerate}
%%%%%%%%%%%%%%%%%%%%%%%%%%%%%%%%%%%%%%%%%%%%%%%%%%%%%%%%%%%%%%%%%%%%%%%%%%%%%%
\newpage%%%%%%%%%%%%%%%%%%%%%%%%%%%%%%%%%%%%%%%%%%%%%%%%%%%%%%%%%%%%%%%%%%%%%%
%%%%%%%%%%%%%%%%%%%%%%%%%%%%%%%%%%%%%%%%%%%%%%%%%%%%%%%%%%%%%%%%%%%%%%%%%%%%%%
\begin{enumerate}
  \item[Problem 4] Non-flat metric: The Friedmann-Robertson-Walker metric for a homogeneous, isotropic universe is given by

  $$
  d s^2=-d t^2+a(t)^2\left[\frac{d r^2}{\left(1-k r^2\right)}+r^2\left(d \theta^2+\sin ^2 \theta d \phi^2\right)\right]
  $$
  
  where the parameter $k$ indicates either a positively curved, spherical universe $(+1)$, a flat, Euclidean universe ( 0 ), or a negatively curved, hyperbolic universe ( -1 ). [You may recognize the $k=1$ metric from its 2 -d form in HW 2.]
  \item[(a)] Which one corresponds to the metric in problem P9.7?
  
  \item[Solution.]
  Problem P9.7 had
  \[
  ds^2 = -dt^2 + dr_{\text{P9.7}}^{\,2} + R^2\sinh^2\!\big(r_{\text{P9.7}}/R\big)\,d\Omega^2,
  \qquad d\Omega^2=d\theta^2+\sin^2\theta\,d\phi^2.
  \]
  if we set $k=-1$ in FRW and define the curvature–radial coordinate $\chi$ by $r_{\text{FRW}}=\sinh\chi$ (so $dr_{\text{FRW}}^2/(1+r_{\text{FRW}}^2)=d\chi^2$), then the FRW spatial metric reads
  \[
  a(t)^2\big[d\chi^2+\sinh^2\chi\,d\Omega^2\big].
  \]
  we can choose a constant scale factor $a(t)\equiv R$ and identifying $\chi=r_{\text{P9.7}}/R$ to reproduce P9.7. 
  Hence P9.7 corresponds to $k=-1$ with $a(t)=R$.

  
  
  \item[(b)] Answer the questions (a-c) from P.9.7 for the metric given here.
  
  \item[Solution.]
  We answer (a–c) from P9.7 for the FRW metric
  \[
  ds^2=-dt^2+a(t)^2\!\left[\frac{dr^2}{1-kr^2}+r^2 d\Omega^2\right]
  \]

  \textbf{(a)}
  Similarly, if we set $dr=d\theta=d\phi=0$, then
  \[
  ds^2=-dt^2 \quad\Longrightarrow\quad d\tau^2\equiv -ds^2=dt^2
  \]
  Thus $\boxed{d\tau=dt}$, the coordinate time $t$ is the proper time.

  \textbf{(b)}
  At fixed $t$, the spatial metric is
  \[
  ds^2 = a(t)^2\!\left[\frac{dr^2}{1-kr^2}+r^2 d\Omega^2\right]
  \]
  which is the metric of a 3–space of constant curvature $k/a(t)^2$. The spacetime should be similarly spherically symmetric (and spatially homogeneous and isotropic). 

  \textbf{(c)}
  The coefficient of $d\Omega^2$ is $a(t)^2 r^2$, so the areal (circumferential) radius is
  \[
  \rho(t,r)=a(t)\,r
  \]
  and the circumference of the equatorial circle ($\theta=\pi/2$) at fixed $t$ is
  \[
  C(t,r)=2\pi\,\rho(t,r)=2\pi\,a(t)\,r
  \]
  The coordinate $r$ is not proper radial distance. The proper radial distance from the origin to coordinate value $r$ at fixed $t$ is
  \[
  \ell_k(t;r)=\int_0^r \sqrt{g_{rr}}\,dr' 
  = a(t)\int_0^r \frac{dr'}{\sqrt{1-k r'^2}}
  \]
  Therefore:
  \[
  \boxed{
  \begin{aligned}
  k=0:&\quad \ell_0(t;r)=a(t)\,r, && C=2\pi\,\ell_0\\[2pt]
  k=+1:&\quad \ell_{+}(t;r)=a(t)\,\sin^{-1} r\quad (0\le r<1), 
  && C=2\pi\,a r \;<\; 2\pi\,\ell_{+}\\[2pt]
  k=-1:&\quad \ell_{-}(t;r)=a(t)\,\sinh^{-1} r,
  && C=2\pi\,a r \;>\; 2\pi\,\ell_{-}
  \end{aligned}}
  \]
  I used $\sin^{-1} r>r$ for $r\in(0,1)$ and $\sinh^{-1} r<r$ for $r>0$. 
  Thus relative to $2\pi\times$proper radius, circles are smaller (closed, $k=+1$), equal (flat, $k=0$), or larger (open, $k=-1$).

    
  \item[(c)] For each value of $k$, find the radial distance from the origin to a given coordinate $r$. Plot the three solutions on one single graph. (Note on integrals: it is ok if you use Mathematica or the like if you don't recognize an integral. If you do use such a tool, please cite it. )
  
  \item[Solution.]
  I used python to plot the three solutions.
  \begin{figure}[H]
    \centering
    \includegraphics[width=0.85\textwidth]{output (1).png}
  \end{figure}

  
  \item[(d)] Suppose there were an object of a known luminosity at a coordinate $r$. Qualitatively, how would its appearance differ at the origin in each of these universes?
  
  \item[Solution.]
  
  Let the source be at coordinate $r$ at the same time $t$ as the observation, then the $\mathbb{S}^2$ at $(t,r)$ has area 
  \[
  A(t,r)=4\pi\,\rho(t,r)^2=4\pi\,\big(a(t)r\big)^2,
  \]
  so the received flux is 
  \[
  F=\frac{L}{4\pi\,\big(a(t)r\big)^2}.
  \]
  Because the areal radius is $a(t)r$ for all $k$ in these curvature coordinates, a fixed coordinate value $r$ gives the same flux and angular size across $k$. 

  However, for sources at the same proper radial distance $\ell$, the appearance differs from $k$:
  \[
  A_k(\ell)=4\pi\,\big(a(t)\,S_k(\ell/a)\big)^2,\quad
  S_{+1}(x)=\sin x,\ \ S_0(x)=x,\ \ S_{-1}(x)=\sinh x
  \]
  Thus, at fixed $\ell$,
  \[
  A_{+1}(\ell)<4\pi\ell^2<A_{-1}(\ell)
  \]
  so objects look brighter and larger in the closed case ($k=+1$) and dimmer and smaller in the open case ($k=-1$), with flat ($k=0$) in between. 
\end{enumerate}
%%%%%%%%%%%%%%%%%%%%%%%%%%%%%%%%%%%%%%%%%%%%%%%%%%%%%%%%%%%%%%%%%%%%%%%%%%%%%%
\newpage%%%%%%%%%%%%%%%%%%%%%%%%%%%%%%%%%%%%%%%%%%%%%%%%%%%%%%%%%%%%%%%%%%%%%%
%%%%%%%%%%%%%%%%%%%%%%%%%%%%%%%%%%%%%%%%%%%%%%%%%%%%%%%%%%%%%%%%%%%%%%%%%%%%%%
\begin{enumerate}
  \item[Problem 5] Time dilation: Consider a satellite in a circular orbit at an altitude of $20,000 \mathrm{~km}$. (The radius of the Earth is $6.4 \times 10^6 \mathrm{~m}$.) Relative to a clock at rest on the surface of the Earth, a clock on the satellite will run slow due to its relative motion (special relativistic time dilation), but fast due to its greater distance from the center of the Earth (gravitational time dilation).

  Which effect is larger? Calculate the amount the clock will gain or lose over the course of one day measured by the clock on Earth $(86,400 \mathrm{sec})$ due to each effect and the total net gain or loss per day.
  
  \item[Solution.]
  Given earth radius $R_\oplus=6.4\times10^{6}\ \mathrm{m}$, and altitude $h=2.0\times10^{7}\ \mathrm{m}$, so orbital radius:
  \[
  r=R_\oplus+h=2.64\times10^{7}\ \mathrm{m}
  \]
  Earth’s standard gravitational parameter:
  \[
  \mu\equiv GM_\oplus=3.986\times10^{14}\ \mathrm{m^3/s^2}
  \]

  \textbf{1)}
  First, we calculate the SR time dilation that slows down the satellite clock.

  For a circular orbit
  \[
  v=\sqrt{\frac{\mu}{r}}
  \quad\Longrightarrow\quad
  v=\sqrt{\frac{3.986\times10^{14}}{2.64\times10^{7}}}
  =3.886\times10^{3}\ \mathrm{m/s}
  \]
  At small $v/c$,
  \[
  \frac{d\tau_{\rm satellite}}{dt}\approx \sqrt{1-\frac{v^2}{c^2}}
  \approx 1-\frac{v^2}{2c^2}
  \]
  Thus the loss per unit time relative to coordinate time is $-\tfrac{v^2}{2c^2}$, and then relative to a day on the Earth surface:

  \[
  \begin{aligned}
  \Delta\tau_{\rm S.R.}\;&\approx\; -\frac{v^2}{2c^2}\,\big(86400 \text{ s})\\
  &=-\frac{(3.886\times10^{3})^2}{2(2.998\times10^{8})^2}(86400)\\
  &=-7.26\times10^{-6}\ \mathrm{s}
  \end{aligned}
  \]
  hence, we have \boxed{-7.26\ \mu\mathrm{s/day}\ } for the SR time dilation.

  \textbf{2)}
  Now we calculate the GR time dilation that speeds up the satellite clock. For a static clock at radius $r$:
  \[
  \frac{d\tau}{dt}=\sqrt{1-\frac{2\mu}{rc^2}}
  \]
  so the rate ratio of satellite static at $r$ vs.\ clock static on the surface at $R_\oplus$ is
  \[
  \frac{d\tau_{\rm stat}(r)}{d\tau_{\rm stat}(R_\oplus)}
  =\sqrt{\frac{1-\dfrac{2\mu}{rc^2}}{1-\dfrac{2\mu}{R_\oplus c^2}}}
  \approx 1+\frac{\mu}{c^2}\!\left(\frac{1}{R_\oplus}-\frac{1}{r}\right)
  \]
  Hence the increase over one Earth-surface day (86400 s) is

  \[
  \begin{aligned}
  \Delta\tau_{\rm GR}
  &= \left[\sqrt{\frac{1-\dfrac{2\mu}{rc^2}}{1-\dfrac{2\mu}{R_\oplus c^2}}}-1\right](86400\,\mathrm{s})
  \approx \frac{\mu}{c^2}\!\left(\frac{1}{R_\oplus}-\frac{1}{r}\right)(86400\,\mathrm{s}) \\
  &= \frac{3.986\times10^{14}}{(2.998\times10^{8})^{2}}
    \left(\frac{1}{6.4\times10^{6}}-\frac{1}{2.64\times10^{7}}\right)
    (86400\,\mathrm{s}) \\
  &= \bigl(5.250\times10^{-10}\bigr)\,(86400\,\mathrm{s})
  = 4.536\times10^{-5}\,\mathrm{s}
  = 45.36\,\mu\mathrm{s}\!/\mathrm{day}
  \end{aligned}
  \]



  \medskip
  \textbf{3)}
  Finally, we see that
  \[
   \Delta\tau_{\rm net}=\Delta\tau_{\rm GR}+\Delta\tau_{\rm SR}
  \;=\;(+45.36-7.26)\ \mu\mathrm{s/day}
  \;=\;+\;38.10\ \mu\mathrm{s/day}\ 
  \]

  \noindent
  Therefore as we have shown, the gravitational speed-up is more than the SR slow-down. The satellite clock gains about $38\ \mu$s per day relative to a clock at rest on Earth’s surface.

\end{enumerate}
%%%%%%%%%%%%%%%%%%%%%%%%%%%%%%%%%%%%%%%%%%%%%%%%%%%%%%%%%%%%%%%%%%%%%%%%%%%%%%
\newpage%%%%%%%%%%%%%%%%%%%%%%%%%%%%%%%%%%%%%%%%%%%%%%%%%%%%%%%%%%%%%%%%%%%%%%





\end{document}
