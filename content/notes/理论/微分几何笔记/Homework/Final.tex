\documentclass[12pt]{article}

% Adjusted page margins to reduce vertical space
\usepackage[top=0.6in, bottom=1in,left=0.8in, right=1.0in]{geometry}

% Formatting and layout packages
\usepackage{titlesec}
\usepackage{lipsum}
% \usepackage{mathpazo} % Font package for aesthetics (switched to Computer Modern)

\usepackage{mdframed}

\newmdenv[
  topline=true, 
  bottomline=true,
  rightline=true, 
  leftline=true, 
  linecolor=black, 
  linewidth=0.8pt,
  backgroundcolor=white,
  innertopmargin=8pt,
  innerbottommargin=8pt,
  skipabove=8pt,
  skipbelow=8pt,
  splitbottomskip=2pt,  % Ensures proper spacing when the box splits across pages
  splittopskip=2pt,
]{solutionbox}

% Auto-wrap any "\item[Solution.]" in a solutionbox until next item at same level
\usepackage{xparse}
\usepackage{xstring}
\usepackage{etoolbox}

% Track nesting depth across list environments
\newcounter{solution@depth}
\newcounter{solution@level}
\newif\ifinsolutionbox
\insolutionboxfalse

% Increase/decrease depth for enumerate/itemize/description
\AtBeginEnvironment{enumerate}{\stepcounter{solution@depth}}
\AtEndEnvironment{enumerate}{%
  % If a solutionbox was opened at this level, close it before exiting
  \ifinsolutionbox
    \ifnum\value{solution@depth}=\value{solution@level}\relax
      \end{solutionbox}%
      \insolutionboxfalse
    \fi
  \fi
  \addtocounter{solution@depth}{-1}%
}
\AtBeginEnvironment{itemize}{\stepcounter{solution@depth}}
\AtEndEnvironment{itemize}{\addtocounter{solution@depth}{-1}}
\AtBeginEnvironment{description}{\stepcounter{solution@depth}}
\AtEndEnvironment{description}{\addtocounter{solution@depth}{-1}}

% Intercept \item to open/close solution boxes based on label
\let\olditem\item
\RenewDocumentCommand{\item}{o}{%
  % If we're starting a new item at the same level as an open solutionbox, close it
  \ifinsolutionbox
    \ifnum\value{solution@depth}=\value{solution@level}\relax
      \end{solutionbox}%
      \insolutionboxfalse
    \fi
  \fi
  % If no optional label, just pass through
  \IfNoValueTF{#1}{%
    \olditem
  }{%
    % If the label is exactly "Solution.", open a new solutionbox
    \IfStrEq{#1}{Solution.}{%
      \olditem[]%
      \begin{solutionbox}%
      \insolutionboxtrue
      \setcounter{solution@level}{\value{solution@depth}}%
    }{%
      \olditem[#1]% other labels unchanged
    }%
  }%
}



% Packages for figures and tables
\usepackage{graphicx}
\usepackage{subcaption}
% Fix for transparent package compatibility issue
\providecommand{\IfPDFManagementActiveF}[1]{#1}
\usepackage{svg}
\usepackage{booktabs}
\usepackage{float}

% Math-related packages
\usepackage{amsmath, amssymb, amsthm}
\usepackage{cancel}

% TikZ for diagrams
\usepackage{tikz}
\usetikzlibrary{arrows,arrows.meta,angles,quotes,calc}

% Declare supported graphic file types
\DeclareGraphicsExtensions{.pdf,.jpg,.tif,.png}

% Theorem and definition environments
\newtheorem{theorem}{Theorem}
\newtheorem{corollary}[theorem]{Corollary}
\newtheorem{lemma}[theorem]{Lemma}
\newtheorem{definition}{Definition}

% Reduce unnecessary vertical spacing
\setlength{\parskip}{0pt}  % No extra space between paragraphs
\setlength{\topsep}{2pt}   % Reduce spacing before lists/theorems
\setlength{\partopsep}{0pt}
\raggedbottom  % Prevents vertical stretching
\pagenumbering{gobble}

% Load hyperref last to apply hidelinks properly
\usepackage[hidelinks]{hyperref}

%%%%%%%%%%%%%%%%%%%%%%%%%%%%%%%%
\begin{document}
\thispagestyle{plain}
\vspace{-4ex}  % Less aggressive spacing reduction


\begin{center}
\begin{tabular}{*{3}{c}}
    \parbox[t]{0.3\linewidth}{\centering\textbf{Final Problem Set}}
    & \parbox[t]{0.3\linewidth}{\centering\textbf{MATH 545\\Differential Geometry}}
    & \parbox[t]{0.3\linewidth}{\centering\textbf{Erise He}}\\[2em]
    \hline
\end{tabular}
\end{center}

\bigskip


\begin{enumerate}
\item[Problem 1] 
\textbf{Show that the helicoid $\sigma(u, t)=(u \cos t, u \sin t, t)$ and the logarithmic cylinder $\sigma(u, t)=$ ($u \cos t, u \sin t, \ln u$) have the same Gaussian curvature function $K(u, t)$, even though these surfaces are not local isometric.}

\item[Part A]
\begin{proof}

  First, we show that the Gaussian curvature $K(u, t)$ is the same for helicoid and logarithmic cylinder.

  \textbf{(1) Helicoid: }
  $$\sigma(u, t)=(u \cos t, u \sin t, t)$$
  we compute the first derivatives, and second derivatives:
  $$\begin{aligned}
  \sigma_u &= (\cos t, \sin t, 0) \\
  \sigma_t &= (-u \sin t, u \cos t, 1)
  \end{aligned}
  \quad \text{and} \quad 
  \begin{aligned}
  \sigma_{uu} &= (0, 0, 0) \\
  \sigma_{ut} &= (-\sin t, \cos t, 0) \\
  \sigma_{tt} &= (-u \cos t, -u \sin t, 0)
  \end{aligned}
  $$
  The \textit{first fundamental form} is:
  $$\begin{aligned}
  g_{uu} &= \langle \sigma_u, \sigma_u \rangle = \cos^2 t + \sin^2 t = 1 \\
  g_{ut} &= \langle \sigma_u, \sigma_t \rangle = - u \cos t \sin t + u \sin t \cos t = 0 \\
  g_{tt} &= \langle \sigma_t, \sigma_t \rangle = u^2 \cos^2 t + u^2 \sin^2 t + 1 = 1 + u^2
  \end{aligned}\Rightarrow \quad g_{ij} = \begin{pmatrix} 1 & 0 \\ 0 & 1 + u^2 \end{pmatrix}$$
  
  Since we know that:
  $$
  \begin{aligned}
  \sigma_u \times \sigma_t & =\operatorname{det}\left(\begin{array}{ccc}
  \mathbf{i} & \mathbf{j} & \mathbf{k} \\
  \cos t & \sin t & 0 \\
  -u \sin t & u \cos t & 1
  \end{array}\right) \\
  & =\mathbf{i}\left|\begin{array}{cc}
  \sin t & 0 \\
  u \cos t & 1
  \end{array}\right|-\mathbf{j}\left|\begin{array}{cc}
  \cos t & 0 \\
  -u \sin t & 1
  \end{array}\right|+\mathbf{k}\left|\begin{array}{cc}
  \cos t & \sin t \\
  -u \sin t & u \cos t
  \end{array}\right| \\
  & =\mathbf{i}(\sin t-0)-\mathbf{j}(\cos t-0)+\mathbf{k}\left(u \cos ^2 t+u \sin ^2 t\right) \\
  & =(\sin t,-\cos t, u)\\
  \left\|\sigma_u \times \sigma_t\right\|& =\sqrt{\sin ^2 t+(-\cos t)^2+u^2}=\sqrt{1+u^2}
  \end{aligned}
  $$
  Then, the normal vector $N(u, t)$ is:
  $$N(u,t) = \frac{\sigma_u \times \sigma_t}{\|\sigma_u \times \sigma_t\|} = \frac{(\sin t,-\cos t, u)}{\sqrt{1+u^2}}$$
  Hence, we have the \textit{second fundamental form}:
  $$
  \begin{aligned}
  & \mathrm{II}_{uu}=\left\langle\sigma_{u u}, N\right\rangle=0 \\
  & \mathrm{II}_{ut}=\left\langle\sigma_{u t}, N\right\rangle=\frac{-\sin ^2 t-\cos ^2 t}{\sqrt{1+u^2}}=-\frac{1}{\sqrt{1+u^2}} \\[-0.2em]
  & \mathrm{II}_{tt}=\left\langle\sigma_{t t}, N\right\rangle=\frac{-u \cos t \sin t+u \sin t \cos t}{\sqrt{1+u^2}}=0
  \end{aligned}
  \quad \Rightarrow \quad \mathrm{II}_{ij} = \begin{pmatrix} 0 & -\frac{1}{\sqrt{1+u^2}} \\ -\frac{1}{\sqrt{1+u^2}} & 0 \end{pmatrix}
  $$
  According to the formula, the shape operator $W$ is:
  $$
  \begin{aligned}
  \left[W^i{ }_j\right]=\left[g^{i k}\right]\left[\mathrm{II}_{k j}\right] & =\left(\begin{array}{cc}
  1 & 0 \\
  0 & \frac{1}{1+u^2}
  \end{array}\right)\left(\begin{array}{cc}
  0 & -\frac{1}{\sqrt{1+u^2}} \\
  -\frac{1}{\sqrt{1+u^2}} & 0
  \end{array}\right) \\
  & =\left(\begin{array}{cc}
  0 & -\frac{1}{\sqrt{1+u^2}} \\
  -\frac{1}{\left(1+u^2\right) \sqrt{1+u^2}} & 0
  \end{array}\right) \\
  & =\left(\begin{array}{cc}
  0 & -\left(1+u^2\right)^{-1 / 2} \\
  -\left(1+u^2\right)^{-3 / 2} & 0
  \end{array}\right)
  \end{aligned}
  $$
  Finally the Gaussian curvature function $K(u, t)$ for helicoid is:
  $$
  K_{\text{hel}} =\operatorname{det}(W) =-\left(-\frac{1}{\sqrt{1+u^2}} \cdot \frac{-1}{\left(1+u^2\right) \sqrt{1+u^2}}\right) =\boxed{-\frac{1}{\left(1+u^2\right)^2}}
  $$
  \textbf{(2) Logarithmic Cylinder: }
  $$\sigma(u, t) = (u \cos t, u \sin t, \ln u)$$
  we compute the first derivatives, and second derivatives:
  $$\begin{aligned}
  \sigma_u &= \left(\cos t, \sin t, \frac{1}{u}\right) \\
  \sigma_t &= (-u \sin t, u \cos t, 0)
  \end{aligned}
  \quad \text{and} \quad 
  \begin{aligned}
  \sigma_{uu} &= \left(0, 0, -\frac{1}{u^2}\right) \\
  \sigma_{ut} &= (-\sin t, \cos t, 0) \\
  \sigma_{tt} &= (-u \cos t, -u \sin t, 0)
  \end{aligned}
  $$
  The \textit{first fundamental form} is:
  $$\begin{aligned}
  g_{uu} &= \langle \sigma_u, \sigma_u \rangle = \cos^2 t + \sin^2 t + \frac{1}{u^2} = 1 + \frac{1}{u^2} = \frac{u^2+1}{u^2} \\
  g_{ut} &= \langle \sigma_u, \sigma_t \rangle = -u \sin t \cos t + u \sin t \cos t + 0 = 0 \\
  g_{tt} &= \langle \sigma_t, \sigma_t \rangle = u^2 \sin^2 t + u^2 \cos^2 t + 0 = u^2
  \end{aligned}\Rightarrow \quad g_{ij} = \begin{pmatrix} \frac{u^2+1}{u^2} & 0 \\ 0 & u^2 \end{pmatrix}$$
  Since we know that:
  $$
  \begin{aligned}
  \sigma_u \times \sigma_t & =\operatorname{det}\left(\begin{array}{ccc}
  \mathbf{i} & \mathbf{j} & \mathbf{k} \\
  \cos t & \sin t & \frac{1}{u} \\
  -u \sin t & u \cos t & 0
  \end{array}\right) \\
  & =\mathbf{i}\left|\begin{array}{cc}
  \sin t & \frac{1}{u} \\
  u \cos t & 0
  \end{array}\right|-\mathbf{j}\left|\begin{array}{cc}
  \cos t & \frac{1}{u} \\
  -u \sin t & 0
  \end{array}\right|+\mathbf{k}\left|\begin{array}{cc}
  \cos t & \sin t \\
  -u \sin t & u \cos t
  \end{array}\right| \\
  & =\mathbf{i}(0-\cos t)-\mathbf{j}(0+\sin t)+\mathbf{k}\left(u \cos^2 t + u \sin^2 t\right) \\
  & =(-\cos t, -\sin t, u)\\
  \left\|\sigma_u \times \sigma_t\right\|& =\sqrt{(-\cos t)^2 + (-\sin t)^2 + u^2} = \sqrt{1+u^2}
  \end{aligned}
  $$
  Then, the normal vector $N(u, t)$ is:
  $$N(u,t) = \frac{\sigma_u \times \sigma_t}{\|\sigma_u \times \sigma_t\|} = \frac{(-\cos t, -\sin t, u)}{\sqrt{1+u^2}}$$
  Hence, we have the \textit{second fundamental form}:
  $$
  \begin{aligned}
  & \mathrm{II}_{uu}=\left\langle\sigma_{uu}, N\right\rangle = \frac{-1/u^2 \cdot u}{\sqrt{1+u^2}} = -\frac{1}{u\sqrt{1+u^2}} \\
  & \mathrm{II}_{ut}=\left\langle\sigma_{ut}, N\right\rangle = \frac{\sin t \cos t - \cos t \sin t}{\sqrt{1+u^2}} = 0 \\
  & \mathrm{II}_{tt}=\left\langle\sigma_{tt}, N\right\rangle = \frac{u \cos^2 t + u \sin^2 t}{\sqrt{1+u^2}} = \frac{u}{\sqrt{1+u^2}}
  \end{aligned}
  \quad \Rightarrow \quad \mathrm{II}_{ij} = \begin{pmatrix} -\frac{1}{u\sqrt{1+u^2}} & 0 \\ 0 & \frac{u}{\sqrt{1+u^2}} \end{pmatrix}
  $$
  
  According to the formula, the shape operator $W$ is:
  $$
  \begin{aligned}
  \left[W^i{ }_j^i\right]=\left[g^{ik}\right]\left[\mathrm{II}_{kj}\right] & = \begin{pmatrix} \frac{u^2}{1+u^2} & 0 \\ 0 & \frac{1}{u^2} \end{pmatrix} \begin{pmatrix} -\frac{1}{u\sqrt{1+u^2}} & 0 \\ 0 & \frac{u}{\sqrt{1+u^2}} \end{pmatrix} \\
  & = \begin{pmatrix} -\frac{u}{(1+u^2)\sqrt{1+u^2}} & 0 \\ 0 & \frac{1}{u\sqrt{1+u^2}} \end{pmatrix} \\
  & = \begin{pmatrix} -u(1+u^2)^{-3/2} & 0 \\ 0 & \frac{1}{u}(1+u^2)^{-1/2} \end{pmatrix}
  \end{aligned}
  $$
  Finally, the Gaussian curvature function $K(u, t)$ for the logarithmic cylinder is:
  $$
  K_{\text{cyl}}= \left(-u(1+u^2)^{-3/2}\right) \cdot \left(\frac{1}{u}(1+u^2)^{-1/2}\right) = \boxed{-\frac{1}{(1+u^2)^2}} = K_{\text{hel}}
  $$
  Therefore, helicoid and logarithmic cylinder have the same Gaussian curvature function $K(u, t)$. 
\end{proof}
\item[Part B]
\begin{proof}

  Now, we show that these surfaces are not local isometric.

  Assume for contradiction there exists a local isometry $\psi: S \rightarrow \tilde{S}$ between helicoid and logarithmic cylinder, that is, $\psi(u, t)=(\tilde{u}(u, t), \tilde{t}(u, t))$. Then, by the definition of local isometry, we should have:
  $$
  \psi^*\tilde g = g
  $$
  From part A, since
  $$
  K_{\text {hel }}(u, t)=-\frac{1}{\left(1+u^2\right)^2} =
  K_{\text {cyl }}(\tilde{u}, \tilde{t})=K_{\text {cyl }}(\psi(u, t))=-\frac{1}{\left(1+\tilde{u}^2\right)^2}
  $$
  we know $u^2 = \tilde{u}^2$. Since $u, \tilde{u} >0$, so $u = \tilde{u}$, thus $du = d\tilde{u}$. From the metric of the logarithmic cylinder,
  $$
  d \tilde{s}^2=\left(1+\frac{1}{\tilde{u}^2}\right) d \tilde{u}^2+\tilde{u}^2 d \tilde{t}^2
  $$
  we can write $d \tilde{t}=\tilde{t}_u d u+\tilde{t}_t d t$, and therefore substitute $\tilde{u}$ and $d\tilde{u}$ to get the pullback:
  $$
  \psi^*\left(d \tilde{s}^2\right)=\left(1+\frac{1}{u^2}\right) d u^2+u^2\left(\tilde{t}_u d u+\tilde{t}_t d t\right)^2
  $$
  Here, the coefficients of $d u^2$ is:
  $$
  \left(\psi^* \tilde{g}\right)_{u u}=\left(1+\frac{1}{u^2}\right)+u^2\left(\tilde{t}_u\right)^2 \geq 1+\frac{1}{u^2}>1
  $$
  However, we have $g_{uu}=1$ for helicoid, so it is impossible to have $\psi^*\tilde g = g$. Hence, the surfaces of helicoid and logarithmic cylinder are not locally isometric.
  
\end{proof}







\end{enumerate}
\newpage
\begin{enumerate}
\item[Problem 2] 

For a vector field $Y$ on a Riemannian manifold, the covariant derivative of $Y$ is a $\binom{1}{1}$ tensor defined by
$$
\nabla Y: Z \mapsto \nabla_Z Y \text {. }
$$
The divergence of $X$ is the trace of this tensor:
$$
\operatorname{div} Y:=\operatorname{tr}(\nabla Y) .
$$
If $Y$ is written as $Y^j \partial_j$ in local coordinates, show that
$$
\operatorname{div} Y=\frac{1}{\sqrt{g}} \partial_j\left(\sqrt{g} Y^j\right)
$$
[You will need Jacobi's matrix derivative formula:
$$
\partial_j \operatorname{det}(A)=\operatorname{det}(A) \operatorname{tr}\left(A^{-1} \partial_j A\right)
$$
assuming $A$ is smooth and invertible.]
\item[Solution] 
\begin{proof}
By the formula for divergence, we have
$$
\begin{aligned}
\operatorname{div} Y=\operatorname{tr}(\nabla Y) &= \nabla_i Y^i \\
&= (\nabla_i(Y^j \partial_j))^i\\
&= (\underbrace{(\partial_iY^j)\partial_j}_{(j\to k)}  + Y^j \underbrace{(\nabla_i\partial_j)}_{= \Gamma_{ij}^k\partial_k})^i\\
&= \left(\left(\partial_i Y^k+\Gamma_{i j}^k Y^j\right) \partial_k\right)^i\\
&= \partial_i Y^i + \Gamma_{i j}^i Y^j\\
\end{aligned}
$$
Now we calculate the christoffel symbols:
$$
\begin{aligned}
  \Gamma_{i j}^i & =\frac{1}{2} g^{i l}\left(\partial_i g_{j l}+\partial_j g_{i l}-\partial_l g_{i j}\right) \\
  & =\frac{1}{2}(\underbrace{g^{i l} \partial_i g_{j l}}_{\cancel{g^{l i} \partial_l g_{j i}}}+g^{i l} \partial_j g_{i l}-\cancel{g^{i l} \partial_l g_{i j}})= \frac{1}{2} g^{i l} \partial_j g_{i l}
\end{aligned}
$$
using symmetry of the metric. We then can rewrite
$$g^{i l} \partial_j g_{i l} = \operatorname{tr}\left(g^{-1} \partial_j g\right)=\frac{1}{g} \partial_j g$$
Let $g = \operatorname{det}(g_{ij})$, then by the given Jacobi's matrix derivative formula:
$$
\partial_j g = g \operatorname{tr}\left(g^{-1} \partial_j g\right)=g g^{i \ell} \partial_j g_{\ell i}=g g^{i \ell} \partial_j g_{i \ell}
$$
Therefore, we have
$$
\Gamma_{i j}^i=\frac{1}{2} (g^{i l} \partial_j g_{i l})=\frac{1}{2} (\frac{1}{g} \partial_j g)=\frac{1}{\sqrt{g}} (\frac{1}{2\sqrt{g}} \partial_j g)=\frac{1}{\sqrt{g}} \partial_j(\sqrt{g})
$$
Finally, we substitute back to get the formula for divergence:
$$
\begin{aligned}
  \operatorname{div} Y & =\partial_i Y^i+\frac{1}{\sqrt{g}} \partial_j(\sqrt{g}) Y^j \\
  & =\frac{1}{\sqrt{g}}\left(\sqrt{g} \partial_j Y^j+\left(\partial_j \sqrt{g}\right) Y^j\right) \\
  & =\boxed{\frac{1}{\sqrt{g}} \partial_j\left(\sqrt{g} Y^j\right) }
  \end{aligned}
$$
\end{proof}


\end{enumerate}

\newpage
\begin{enumerate}
\item[Problem 3]
Let ($M, g$) be a two-dimensional Riemannian manifold, and define a new metric by setting
$$
\tilde{g}=e^{2 \phi} g
$$
for some function $\phi \in C^{\infty}(M)$. (The metric $\tilde{g}$ is said to be \textit{conformal} to $g$.) Show that the respective Gaussian curvatures $K$ and $\tilde{K}$ are related by
$$
\tilde{K}=e^{-2 \phi}(K-\Delta \phi),
$$
where $\Delta$ is the Laplacian associated to $g$
$$
\Delta \phi:=\operatorname{div}(\nabla \phi)
$$

\item[Solution] 
\begin{proof}
We start with the formula for 
  $$
  K=\frac{1}{2\sqrt{g}} \partial_i \left(\sqrt{g} \Gamma_{i j}^j\right)
  $$
  
\end{proof}
\end{enumerate}

%%%%%%%%%%%%%%%%%%%%%%%%%%%%%%
\end{document}
