\documentclass[11pt,a4paper]{article}

% --- Essential Packages ---
\usepackage{amsmath, amssymb, amsthm}
\usepackage{physics}
\usepackage{geometry}
\usepackage{fancyhdr}
\usepackage{cancel}
\usepackage{xcolor}
\usepackage[many]{tcolorbox}
\usepackage{enumitem}
\usepackage{tikz}
\usepackage{graphicx}

% --- Page Layout ---
\geometry{top=0.7in, bottom=0.7in, left=0.7in, right=0.7in}
\setlength{\parindent}{0pt}
\setlength{\parskip}{0.5em}

% --- Custom Colors ---
\definecolor{graytext}{gray}{0.5}

% --- Custom Commands ---
\newcommand{\gray}[1]{{\color{graytext}#1}}
\newcommand{\examnote}[1]{{\color{red}\footnotesize\textbf{[Exam Note: #1]}}}

% --- Header Configuration ---
\newcommand{\courseName}{PHYS 458}
\newcommand{\hwNumber}{HW 7}
\newcommand{\hwTopic}{Oppenheimer–Volkoff, Eddington–Finkelstein, Hawking Radiation}

\pagestyle{fancy}
\fancyhf{}
\lhead{\textbf{\courseName: \hwNumber}}
\rhead{\hwTopic}
\cfoot{\thepage}

% --- Solution Box Environment ---
\newtcolorbox{solutionbox}{
    breakable,
    enhanced,
    colback=white,
    colframe=black,
    title=\textbf{Derivation Transcript},
    fonttitle=\bfseries\small,
    arc=0mm,
    boxrule=1pt,
    left=5mm, right=5mm, top=5mm, bottom=5mm,
    before skip=10pt,
    after skip=20pt
}

\newtcolorbox{problembox}{
    colback=gray!10!white,
    colframe=gray!60!black,
    title=\textbf{Problem Statement},
    fonttitle=\bfseries\small,
    arc=0mm,
    boxrule=1pt,
    left=5mm, right=5mm, top=5mm, bottom=5mm
}

\title{\vspace{-2cm} \courseName: \hwNumber\ Solutions}
\author{}
\date{}

\begin{document}

%=========================================================
\section*{1.\ Oppenheimer–Volkoff Equation (Book Problem 23.5)}

%-------------------- (a) --------------------
\subsection*{(a) Show that for a perfect fluid $T_{\mu\nu}-\tfrac12 g_{\mu\nu}T$ has the stated form}

\begin{problembox}
For a perfect fluid,
\[
T_{\mu\nu} = (\rho + P) u_\mu u_\nu + g_{\mu\nu} P
\]
and
\[
T \equiv g^{\mu\nu} T_{\mu\nu}
\]
Show that
\[
T_{\mu\nu} - \tfrac12 g_{\mu\nu} T
\]
can be written in the form used in the text.
\end{problembox}

\begin{solutionbox}
We start from
\[
T_{\mu\nu} = (\rho + P) u_\mu u_\nu + g_{\mu\nu} P
\]

Compute the trace
$$
\begin{aligned}
T
&= g^{\mu\nu} T_{\mu\nu} \\
&= (\rho + P) g^{\mu\nu} u_\mu u_\nu + g^{\mu\nu} g_{\mu\nu} P \\
&= (\rho + P)\underbrace{\left(\vec{u}\cdot\vec{u}\right)}_{=-1} + 4P \\
&= -\rho + 3P
\end{aligned}
$$

Then
$$
\begin{aligned}
T_{\mu\nu} - \frac12 g_{\mu\nu} T
&= (\rho + P) u_\mu u_\nu + g_{\mu\nu} P - \frac12 g_{\mu\nu}(-\rho + 3P) \\
&= (\rho + P) u_\mu u_\nu + \frac12 g_{\mu\nu}(\rho - P)
\end{aligned}
$$
\end{solutionbox}

%-------------------- (b) --------------------
\subsection*{(b) 4-velocity for fluid at rest}

\begin{problembox}
Assume the star's interior is static and spherically symmetric with metric
\[
ds^2 = -A(r)\,dt^2 + B(r)\,dr^2 + r^2(d\theta^2 + \sin^2\theta\,d\phi^2)
\]
For a perfect fluid at rest in these coordinates, write down the 4-velocity $u^\mu$ and its covector $u_\mu$.
\end{problembox}

\begin{solutionbox}
We consider a fluid at rest in these coordinates, so its 4-velocity is
\[
u^\mu =
\begin{bmatrix}
u^t \\[4pt]
0 \\[4pt]
0 \\[4pt]
0
\end{bmatrix}
\qquad
\text{(for fluid at rest)}
\]

The covector is
\[
u_\mu = g_{\mu\nu} u^\nu
\]

For the assumed diagonal metric,
\[
u_t = g_{tt} u^t \,, \qquad u_i = 0
\]

Normalize using $\vec{u}\cdot\vec{u} = -1$:
$$
\begin{aligned}
-1 &= g_{\mu\nu} u^\mu u^\nu = g_{tt} (u^t)^2 = -A (u^t)^2 \\
\Rightarrow\quad u^t &= \frac{1}{\sqrt{A}}
\end{aligned}
$$

Then
\[
u_t = g_{tt} u^t = -A \frac{1}{\sqrt{A}} = -\sqrt{A}
\]
\end{solutionbox}

%-------------------- (c) --------------------
\subsection*{(c) Show that the Einstein equation implies Eqs.\ (23.44a–c)}

\begin{problembox}
Using the above perfect-fluid form and 4-velocity, show that
\[
R_{tt} = 4\pi G A(\rho + 3P), \qquad
R_{rr} = 4\pi G B(\rho - P), \qquad
R_{\theta\theta} = 4\pi G r^2 (\rho - P)
\]
\end{problembox}

\begin{solutionbox}
Einstein equations (in these units) are
\[
R_{\mu\nu} = 8\pi G\left(T_{\mu\nu} - \frac12 g_{\mu\nu} T\right)
\]

Using the result from part (a),
\[
T_{\mu\nu} - \frac12 g_{\mu\nu} T
= (\rho + P) u_\mu u_\nu + \frac12 g_{\mu\nu}(\rho - P)
\]

\textbf{$R_{tt}$:}
$$
\begin{aligned}
R_{tt}
&= 8\pi G\left[T_{tt} - \frac12 g_{tt} T\right] \\
&= 8\pi G\left[(\rho + P) (u_t)^2 + \frac12 g_{tt}(\rho - P)\right] \\
&= 8\pi G\left[(\rho + P)A - \frac12 A(\rho - P)\right] \\
&= 8\pi G\,A\left[\rho + P - \frac12\rho + \frac12 P\right] \\
&= 4\pi G\,A(\rho + 3P)
\end{aligned}
$$

\textbf{$R_{rr}$:}
The $u_r$ component vanishes, so
$$
\begin{aligned}
R_{rr}
&= 8\pi G\left[0 + \frac12 g_{rr}(\rho - P)\right] \\
&= 4\pi G\,B(\rho - P)
\end{aligned}
$$

\textbf{$R_{\theta\theta}$:}
Similarly,
$$
\begin{aligned}
R_{\theta\theta}
&= 8\pi G\left[0 + \frac12 g_{\theta\theta}(\rho - P)\right] \\
&= 4\pi G\,r^2(\rho - P)
\end{aligned}
$$
\end{solutionbox}

%-------------------- (d) --------------------
\subsection*{(d) Derive Eq.\ (23.46): combination of $R_{tt}$, $R_{rr}$, $R_{\theta\theta}$}

\begin{problembox}
Using Eqs.\ (23.6) for the diagonal metric (with all time derivatives set to zero),
show that
\[
\frac{R_{tt}}{A} + \frac{R_{rr}}{B} + \frac{2R_{\theta\theta}}{r^2}
= 16\pi G\rho
\]
and hence that
\[
\frac{d}{dr}\left[r\left(1 - \frac1B\right)\right] = 8\pi G\rho\,r^2
\]
\end{problembox}

\begin{solutionbox}
Starting from Eq.\ (23.6) in the text and setting all time-derivatives to zero,
\[
\frac{R_{tt}}{A} + \frac{R_{rr}}{B} + \frac{2R_{\theta\theta}}{r^2}
\]
is written as
$$
\begin{aligned}
\frac{R_{tt}}{A} + \frac{R_{rr}}{B} + \frac{2R_{\theta\theta}}{r^2}
&=
\frac{1}{2AB}
\Bigg[
\frac{d^2A}{dr^2}
-\frac{1}{2A}\left(\frac{dA}{dr}\right)^2
+\frac{1}{2B}\frac{dA}{dr}\frac{dB}{dr}
+2\frac{dA}{dr}\frac1r
\Bigg] \\[4pt]
&\quad
+
\frac{1}{2AB}
\Bigg[
\frac{d^2B}{dr^2}
-\frac{1}{2B}\left(\frac{dB}{dr}\right)^2
+\frac{1}{2A}\frac{dA}{dr}\frac{dB}{dr}
-2\frac{dB}{dr}\frac1r
\Bigg] \\[4pt]
&\quad
+\frac{2}{r^2}
\Bigg[
-\frac{r}{2AB}\frac{dA}{dr}
+\frac{r}{2B^2}\frac{dB}{dr}
+1-\frac1B
\Bigg]
\end{aligned}
$$

Many terms cancel between the first two big brackets and with pieces of the $\theta\theta$ term; symbolically
$$
\begin{aligned}
\frac{R_{tt}}{A} + \frac{R_{rr}}{B} + \frac{2R_{\theta\theta}}{r^2}
&=
\frac{2}{B^2}\frac{dB}{dr}
+\frac{2}{r^2}
-\frac{2}{r^2 B} \\
&=
\frac{2}{r^2}
\left[
r\frac{d}{dr}\left(1-\frac1B\right)
+ \left(1-\frac1B\right)
\right] \\
&=
\frac{2}{r^2}\frac{d}{dr}\Bigl[r\left(1-\frac1B\right)\Bigr]
\end{aligned}
$$

By the Einstein equation, this combination must equal $16\pi G\rho$, so
\[
\frac{2}{r^2}\frac{d}{dr}\Bigl[r\left(1-\frac1B\right)\Bigr]
= 16\pi G\rho
\]

Thus
\[
\frac{d}{dr}\left[r\left(1-\frac1B\right)\right]
= 8\pi G\rho\,r^2
\]
which is Eq.\ (23.46).
\end{solutionbox}

%-------------------- (e) --------------------
\subsection*{(e) Integrate Eq.\ (23.46) to obtain $B(r)$}

\begin{problembox}
Integrate
\[
\frac{d}{dr}\left[r\left(1 - \frac1B\right)\right]
= 8\pi G\rho\,r^2
\]
to obtain Eqs.\ (23.47a–b):
\[
B = \left[1 - \frac{2Gm(r)}{r}\right]^{-1},
\qquad
m(r) \equiv \int_0^r 4\pi\rho r^2\,dr
\]
and interpret $m(r)$.
\end{problembox}

\begin{solutionbox}
Starting from
\[
\frac{d}{dr}\left[r\left(1 - \frac1B\right)\right]
= 8\pi G\rho\,r^2
\]

Integrate from $0$ to $r$:
$$
\begin{aligned}
r\left(1 - \frac1B\right)
&= 2G m(r)
\end{aligned}
$$
where
\[
m(r) \equiv \int_0^r 4\pi\rho r^2\,dr
\]

Thus
\[
B = \left[1 - \frac{2Gm(r)}{r}\right]^{-1}
\]

For a star of radius $R$, $m(R) = M$ is the Schwarzschild mass.
\end{solutionbox}

%-------------------- (f) --------------------
\subsection*{(f) Use $\nabla_\nu T^{\mu\nu}=0$ to find $A(r)$}

\begin{problembox}
Use the energy–momentum conservation equation
\[
\nabla_\nu T^{\mu\nu} = 0
\]
and take the $\mu = r$ component to derive Eq.\ (23.48):
\[
\frac{1}{A}\frac{dA}{dr} = -\frac{2}{\rho + P}\frac{dP}{dr}
\]
\end{problembox}

\begin{solutionbox}
Take the $\mu = r$ component:
\[
\nabla_\nu T^{r\nu} = \partial_\nu T^{r\nu}
+ \Gamma^{r}_{\nu\alpha} T^{\alpha\nu}
+ \Gamma^{\nu}_{\nu\alpha} T^{r\alpha} = 0
\]

For a perfect fluid
\[
T^{\mu\nu} = (\rho + P) u^\mu u^\nu + g^{\mu\nu} P
\]

In the static, spherically symmetric case, $u^r=0$ and $\rho=\rho(r)$, $P=P(r)$, so after simplification only the $tt$-terms survive. Using $u^t = 1/\sqrt{A}$ and $u_t=-\sqrt{A}$ one finds
$$
\begin{aligned}
\nabla_\nu T^{r\nu}
&=
\Gamma^{r}_{tt} (\rho+P)(u^t)^2
+ g^{rr} \frac{dP}{dr}
\end{aligned}
$$

From the diagonal metric worksheet,
\[
\Gamma^{t}_{tr} = \frac{1}{2A}\frac{dA}{dr}
\,,\qquad
\Gamma^{r}_{tt} = \frac{1}{2B}\frac{dA}{dr}
\]

Also $g^{rr} = 1/B$. Thus
$$
\begin{aligned}
0
&= \Gamma^{r}_{tt} (\rho+P)(u^t)^2 + g^{rr}\frac{dP}{dr} \\
&= \frac{1}{2B}\frac{dA}{dr} (\rho+P)\frac{1}{A}
+ \frac{1}{B}\frac{dP}{dr}
\end{aligned}
$$

Multiply by $B$:
\[
0 = \frac{1}{2A}(\rho+P)\frac{dA}{dr} + \frac{dP}{dr}
\]

Hence
\[
\frac{1}{A}\frac{dA}{dr} = -\frac{2}{\rho + P}\frac{dP}{dr}
\]
which is Eq.\ (23.48).
\end{solutionbox}

%-------------------- (g) --------------------
\subsection*{(g) Derive the Oppenheimer–Volkoff equation (23.49)}

\begin{problembox}
Use Eqs.\ (23.6c), (23.44c),
\[
R_{\theta\theta}
= -\frac{r}{2AB}\frac{dA}{dr}
+ \frac{r}{2B^2}\frac{dB}{dr}
+ 1 - \frac1B
\]
\[
R_{\theta\theta} = 4\pi G r^2(\rho - P)
\]
together with
\[
B = \left[1 - \frac{2Gm(r)}{r}\right]^{-1},
\qquad
\frac{1}{A}\frac{dA}{dr} = -\frac{2}{\rho + P}\frac{dP}{dr}
\]
to show that
\[
\frac{dP}{dr}
= -\frac{\rho + P}{r^2}
\frac{4\pi G r^3 P + Gm(r)}{1 - 2Gm(r)/r}
\]
which is Eq.\ (23.49), the Oppenheimer–Volkoff equation.
\end{problembox}

\begin{solutionbox}
Starting from Eq.\ (23.6c),
\[
R_{\theta\theta}
= -\frac{r}{2AB}\frac{dA}{dr}
+ \frac{r}{2B^2}\frac{dB}{dr}
+ 1 - \frac1B
\]

By Eq.\ (23.44c),
\[
R_{\theta\theta} = 4\pi G r^2(\rho - P)
\]

Substitute $B^{-1} = 1 - 2Gm(r)/r$ and Eq.\ (23.48) for $dA/dr$.
After algebra (as in the notes) one arrives at
$$
\begin{aligned}
4\pi G r^2 (\rho - P)
&= \frac{r}{2}\frac{1}{\rho + P}\frac{dP}{dr}
+ \frac{Gm(r)}{r}
+ 2\pi G r^2(\rho + P)
\end{aligned}
$$
which rearranges to
\[
\frac{dP}{dr}
= -\frac{\rho + P}{r^2}
\frac{4\pi G r^3 P + Gm(r)}{1 - 2Gm(r)/r}
\]

This is Eq.\ (23.49), the Oppenheimer–Volkoff equation for stellar structure.
\end{solutionbox}

%=========================================================
\section*{2.\ Application of Eq.\ (23.49) to Gravitational Collapse}

\begin{problembox}
Show that Eq.\ (23.49) reduces in the Newtonian limit to
\[
\frac{dP}{dr} = -\frac{GM(r)\rho}{r^2}
\]
Discuss qualitatively what happens to Eq.\ (23.49) when $r<2GM$ and show that this is consistent with the neutron star collapsing to a point mass.
\end{problembox}

\begin{solutionbox}
Starting from the Oppenheimer–Volkoff equation
\[
\frac{dP}{dr}
= -\frac{\rho + P}{r^2}
\frac{4\pi G r^3 P + Gm(r)}{1 - 2Gm(r)/r}
\]

In the Newtonian limit,
\[
P \ll \rho, \qquad 2Gm(r) \ll r
\]

so
\[
\rho + P \approx \rho,
\qquad
1 - \frac{2Gm(r)}{r} \approx 1,
\qquad
4\pi G r^3 P \ll Gm(r)
\]

Hence
\[
\frac{dP}{dr}
\approx -\rho\,\frac{Gm(r)}{r^2}
\]
which is the familiar Newtonian hydrostatic equilibrium equation.

A sketch of $P(r)$ shows $dP/dr < 0$; pressure increases toward the center to support the outer layers.

When mass is contained within $r < 2GM$, the factor $(1 - 2Gm(r)/r)$ changes sign. In this regime,
\[
\frac{dP}{dr} > 0
\]
and the interior pressure decreases as one goes inward; gravitational collapse is then inevitable.
\end{solutionbox}

%=========================================================
\section*{3.\ Eddington–Finkelstein Coordinates}

%-------------------- 3(a) --------------------
\subsection*{(a) Ingoing photon geodesic $t(r)$}

\begin{problembox}
Using the Schwarzschild metric, integrate the radial equation of motion for an ingoing photon to find $t$ as a function of $r$.
\end{problembox}

\begin{solutionbox}
From Table 14.1 (or Ch.\ 12) for null geodesics in Schwarzschild,
\[
\frac{dr}{dt}
= \pm\left(1 - \frac{2GM}{r}\right)
\sqrt{1 - \left(1 - \frac{2GM}{r}\right)\frac{b^2}{r^2}}
\]

For a radially ingoing (or outgoing) photon we take $b=0$, so
\[
\frac{dr}{dt} = -\left(1 - \frac{2GM}{r}\right)
\]

Thus
\[
\frac{dt}{dr} = -\frac{1}{1 - 2GM/r}
\]

Integrate:
$$
\begin{aligned}
t
&= -\int \frac{dr}{1 - 2GM/r} \\
&= -\left[r + 2GM \ln(r - 2GM)\right] + \text{const.}
\end{aligned}
$$
\end{solutionbox}

%-------------------- 3(b) --------------------
\subsection*{(b) Coordinate transformation to ingoing Eddington–Finkelstein time}

\begin{problembox}
Show that the coordinate transformation
\[
t \to \bar{t} = t + 2GM\ln(r - 2GM)
\]
puts the ingoing photon trajectory from part (a) on a straight line at $45^\circ$ to the $r$-axis in the $(t,r)$–plane.
\end{problembox}

\begin{solutionbox}
Define the new time coordinate
\[
\bar{t} = t + 2GM\ln(r - 2GM)
\]

Using the result from part (a),
\[
t = -\left(r + 2GM \ln(r - 2GM)\right) + \text{const.}
\]

Then
\[
\bar{t} = -r + \text{const.}
\]

Thus in the $(\bar{t}, r)$ diagram, the ingoing photon follows
\[
\bar{t} + r = \text{const.}
\]
which is a straight line at $45^\circ$.
\end{solutionbox}

%-------------------- 3(c) --------------------
\subsection*{(c) Schwarzschild metric in ingoing Eddington–Finkelstein form}

\begin{problembox}
Write the Schwarzschild metric in ingoing Eddington–Finkelstein coordinates and show that it is well behaved at $r = 2GM$. Sketch the light cones.
\end{problembox}

\begin{solutionbox}
We have
\[
\bar{t} = t + 2GM \ln(r - 2GM)
\quad\Rightarrow\quad
d\bar{t} = dt + \frac{2GM}{r - 2GM}\,dr
\]

Solve for $dt$:
\[
dt = d\bar{t} - \frac{2GM}{r - 2GM}\,dr
\]

Plug this into the Schwarzschild metric
\[
ds^2 = -\left(1 - \frac{2GM}{r}\right)dt^2
+ \left(1 - \frac{2GM}{r}\right)^{-1}dr^2 + r^2 d\Omega^2
\]

After algebra, this gives
\[
ds^2 = -\left(1 - \frac{2GM}{r}\right)d\bar{t}^2
+ \frac{4GM}{r}\,d\bar{t}\,dr
+ \left(1 + \frac{2GM}{r}\right)dr^2
+ r^2 d\Omega^2
\]

This form of the metric is regular at $r = 2GM$. The associated spacetime diagram shows ingoing null geodesics as straight lines at $45^\circ$, while outgoing null geodesics are tilted and ``trapped'' as they cross $r=2GM$ toward the singularity.
\end{solutionbox}

%=========================================================
\section*{4.\ Hawking Radiation and Mini Black Holes}

\begin{problembox}
The most energetic particle ever measured was a high-energy cosmic ray proton with energy $E \sim 3\times10^{20}\,\mathrm{eV}$. Suppose such a particle were to interact with a particle in the upper atmosphere, converting all of its energy into a black hole with the equivalent mass. How long would it live? Should we worry about a rain of mini–black holes falling to Earth and destroying us all?
\end{problembox}

\begin{solutionbox}
The lifetime of a black hole evaporating via Hawking radiation is given in Eq.\ (16.9):
\[
\tau_{\text{life}} = 2.095\times10^{67}\,\text{yr}
\left(\frac{M}{M_\odot}\right)^3
\]

Convert $1\,\mathrm{eV}$ to solar masses:
\[
1\,\mathrm{eV}
= 1.782\times10^{-36}\,\mathrm{kg}
\times\frac{1M_\odot}{1.989\times10^{30}\,\mathrm{kg}}
= 8.96\times10^{-67} M_\odot
\]

The ``OMG'' particle had $E \sim 3\times10^{20}\,\mathrm{eV}$ so
\[
M \sim 2.7\times10^{-46} M_\odot
\]

Plug into the lifetime:
\[
\tau_{\text{life}} \sim 4\times10^{-70}\,\text{yr}
\sim 10^{-62}\,\text{s}
\]

Conclusion: such a mini black hole would evaporate essentially instantaneously. We'll be fine.
\end{solutionbox}

\end{document}
