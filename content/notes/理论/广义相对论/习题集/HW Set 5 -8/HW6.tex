\documentclass[11pt,a4paper]{article}

% --- Essential Packages ---
\usepackage{amsmath, amssymb, amsthm}
\usepackage{physics}
\usepackage{geometry}
\usepackage{fancyhdr}
\usepackage{cancel}
\usepackage{xcolor}
\usepackage[many]{tcolorbox}
\usepackage{enumitem}
\usepackage{tikz}
\usepackage{graphicx}

% --- Page Layout ---
\geometry{top=0.7in, bottom=0.7in, left=0.7in, right=0.7in}
\setlength{\parindent}{0pt}
\setlength{\parskip}{0.5em}

% --- Custom Colors ---
\definecolor{graytext}{gray}{0.5}

% --- Custom Commands ---
\newcommand{\gray}[1]{{\color{graytext}#1}}
\newcommand{\examnote}[1]{{\color{red}\footnotesize\textbf{[Exam Note: #1]}}}

% --- Header Configuration ---
\newcommand{\courseName}{PHYS 458}
\newcommand{\hwNumber}{HW 6}
\newcommand{\hwTopic}{Morris--Thorne Wormhole}

\pagestyle{fancy}
\fancyhf{}
\lhead{\textbf{\courseName: \hwNumber}}
\rhead{\hwTopic}
\cfoot{\thepage}

% --- Boxes ---
\newtcolorbox{solutionbox}{
    breakable,
    enhanced,
    colback=white,
    colframe=black,
    title=\textbf{Derivation Transcript},
    fonttitle=\bfseries\small,
    arc=0mm,
    boxrule=1pt,
    left=5mm, right=5mm, top=5mm, bottom=5mm,
    before skip=10pt,
    after skip=20pt
}

\newtcolorbox{problembox}{
    colback=gray!10!white,
    colframe=gray!60!black,
    title=\textbf{Problem Statement},
    fonttitle=\bfseries\small,
    arc=0mm,
    boxrule=1pt,
    left=5mm, right=5mm, top=5mm, bottom=5mm
}

\title{\vspace{-2cm} \courseName: \hwNumber\ Solutions}
\author{}
\date{}

\begin{document}

\section*{1. Morris--Thorne Wormhole}

The metric is
\[
ds^2 = -dt^2 + dr^2 + (b^2 + r^2)\left(d\theta^2 + \sin^2\theta\, d\phi^2\right)
\]
with constant $b$.  Non-zero Christoffel symbols used below include
\[
\Gamma^r_{\theta\theta} = -r, \qquad
\Gamma^r_{\phi\phi} = -r\sin^2\theta, \qquad
\Gamma^\theta_{\phi\phi} = -\sin\theta\cos\theta, \qquad
\Gamma^\phi_{\theta\phi} = \frac{\cos\theta}{\sin\theta}, \qquad
\Gamma^\phi_{r\phi} = \Gamma^\theta_{r\theta} = \frac{r}{b^2 + r^2}
\]

% ---------- (a) ----------
\subsection*{(a) Spherical symmetry and meaning of $r$}

\begin{problembox}
Is this metric spherically symmetric?  What does the coordinate $r$ measure?
\end{problembox}

\begin{solutionbox}
Yes.

For $dt = 0$, $dr = 0$,
\[
ds^2 = (b^2 + r^2)\left(d\theta^2 + \sin^2\theta\,d\phi^2\right)
\]
which is the metric for the surface of a sphere of radius $\sqrt{b^2 + r^2}$

$t$ measures proper time, and $r$ measures radial distance.
\examnote{Remember: spheres are at constant $t$ and $r$ with area $4\pi(b^2 + r^2)$.}
\end{solutionbox}

% ---------- (b) ----------
\subsection*{(b) Computing Christoffel symbols}

\begin{problembox}
Calculate one of the non-zero $\Gamma^r_{\mu\nu}$, and either $\Gamma^\phi_{r\phi}$ or $\Gamma^\theta_{r\theta}$.
\end{problembox}

\begin{solutionbox}
Start from the geodesic equation
\[
0 = \frac{d^2 x^\mu}{d\tau^2} + \Gamma^\mu_{\alpha\beta} u^\alpha u^\beta
\]
which can be written as
\[
0 = \frac{d}{d\tau}\!\left(g_{\mu\nu}\frac{dx^\nu}{d\tau}\right)
    - \frac{1}{2}\,\partial_\mu g_{\alpha\beta}\,u^\alpha u^\beta
\]

\textbf{Take $\mu = r$:}
\[
\frac{d^2 r}{d\tau^2} + \Gamma^r_{\alpha\beta} u^\alpha u^\beta = 0
\]
Using the second form,
\[
\frac{d}{d\tau}\!\left(g_{rr}\frac{dr}{d\tau}\right)
      - \frac{1}{2}\,\partial_r g_{\alpha\beta}\,u^\alpha u^\beta = 0
\]
Since $g_{rr} = 1$,
\[
\frac{d^2 r}{d\tau^2}
  - \frac{1}{2}\,\partial_r g_{\theta\theta}(u^\theta)^2
  - \frac{1}{2}\,\partial_r g_{\phi\phi}(u^\phi)^2 = 0
\]

For $\Gamma^r_{\theta\theta}$:
\[
-\frac{1}{2}\,\partial_r(b^2 + r^2) = -r
\qquad\Rightarrow\qquad
\Gamma^r_{\theta\theta} = -r
\]

For $\Gamma^r_{\phi\phi}$:
\[
-\frac{1}{2}\,\partial_r\!\left((b^2 + r^2)\sin^2\theta\right)
  = -r\sin^2\theta
\qquad\Rightarrow\qquad
\Gamma^r_{\phi\phi} = -r\sin^2\theta
\]

\bigskip

Now list the metric components explicitly (Morris--Thorne metric):
\[
g_{tt} = -1, \qquad
g_{rr} = 1, \qquad
g_{\theta\theta} = b^2 + r^2, \qquad
g_{\phi\phi} = (b^2 + r^2)\sin^2\theta
\]

\textbf{Use the geodesic equation to find $\Gamma^\theta_{r\theta}$.}

Take $\mu = \theta$:
\[
0 = \frac{d^2\theta}{d\tau^2} + \Gamma^\theta_{\alpha\beta} u^\alpha u^\beta
\]

Writing this from the form
\[
0 = \frac{d}{d\tau}\!\left(g_{\theta\nu}\frac{dx^\nu}{d\tau}\right)
    - \frac{1}{2}\,\partial_\theta g_{\alpha\beta}\,u^\alpha u^\beta
\]
and keeping the non-zero pieces:
\[
g_{\theta\theta}\frac{d^2\theta}{d\tau^2}
   + \partial_r g_{\theta\theta}\,\frac{dr}{d\tau}\frac{d\theta}{d\tau}
   - \frac{1}{2}\,\partial_\theta g_{\phi\phi}\left(\frac{d\phi}{d\tau}\right)^2
   = 0
\]

The middle term is
\[
\partial_r g_{\theta\theta}\,\frac{dr}{d\tau}\frac{d\theta}{d\tau}
 = 2r\,\frac{dr}{d\tau}\frac{d\theta}{d\tau}
\]
Comparing with the geodesic equation shows
\[
\Gamma^\theta_{r\theta} = \Gamma^\theta_{\theta r}
   = \frac{1}{2}\frac{\partial_r g_{\theta\theta}}{g_{\theta\theta}}
   = \frac{r}{b^2 + r^2}
\]

(From the last term one similarly reads off $\Gamma^\theta_{\phi\phi} = -\sin\theta\cos\theta$.)

\bigskip

\textbf{Use the geodesic equation to find $\Gamma^\phi_{r\phi}$.}

Take $\mu = \phi$:
\[
0 = \frac{d^2\phi}{d\tau^2} + \Gamma^\phi_{\alpha\beta} u^\alpha u^\beta
\]

Again from
\[
0 = \frac{d}{d\tau}\!\left(g_{\phi\nu}\frac{dx^\nu}{d\tau}\right)
    - \frac{1}{2}\,\partial_\phi g_{\alpha\beta}\,u^\alpha u^\beta
\]
the non-zero pieces give
\[
g_{\phi\phi}\frac{d^2\phi}{d\tau^2}
  + \partial_r g_{\phi\phi}\,\frac{dr}{d\tau}\frac{d\phi}{d\tau}
  + \partial_\theta g_{\phi\phi}\,\frac{d\theta}{d\tau}\frac{d\phi}{d\tau}
  = 0
\]

From the $r\phi$ term,
\[
\Gamma^\phi_{r\phi} = \Gamma^\phi_{\phi r}
  = \frac{1}{2}\frac{\partial_r g_{\phi\phi}}{g_{\phi\phi}}
  = \frac{2r\sin^2\theta}{2(b^2 + r^2)\sin^2\theta}
  = \frac{r}{b^2 + r^2}
\]

(The last term gives $\Gamma^\phi_{\theta\phi} = \cot\theta$.)
\examnote{For quick reference: $\Gamma^r_{\theta\theta}=-r$, $\Gamma^r_{\phi\phi}=-r\sin^2\theta$, and $\Gamma^\theta_{r\theta}=\Gamma^\phi_{r\phi}=r/(b^2+r^2)$.}
\end{solutionbox}

% ---------- (c) ----------
\subsection*{(c) Geodesic equations for $t$, $r$, and $\phi$}

\begin{problembox}
Write down the geodesic equations for $t$, $r$, and $\phi$.  You may use the symmetry of the metric to set $\theta = \pi/2$.
\end{problembox}

\begin{solutionbox}
Start from
\[
0 = \frac{d^2 x^\mu}{d\tau^2} + \Gamma^\mu_{\alpha\beta}
      \frac{dx^\alpha}{d\tau}\frac{dx^\beta}{d\tau}
\]

\textbf{$\mu = t$:}

All $\Gamma^t_{\mu\nu} = 0$, so
\[
\frac{d^2 t}{d\tau^2} = 0
\]

\textbf{$\mu = r$:}

\[
0 = \frac{d^2 r}{d\tau^2}
  + \Gamma^r_{\theta\theta}\left(\frac{d\theta}{d\tau}\right)^2
  + \Gamma^r_{\phi\phi}\left(\frac{d\phi}{d\tau}\right)^2
\]

Using $\Gamma^r_{\theta\theta} = -r$ and
$\Gamma^r_{\phi\phi} = -r\sin^2\theta$:
\[
0 = \frac{d^2 r}{d\tau^2}
  - r\left(\frac{d\theta}{d\tau}\right)^2
  - r\sin^2\theta\left(\frac{d\phi}{d\tau}\right)^2
\]

Set $\theta = \pi/2 = \text{const.}$, so $d\theta/d\tau = 0$ and $\sin^2\theta = 1$:
\[
\frac{d^2 r}{d\tau^2}
 = r\left(\frac{d\phi}{d\tau}\right)^2
\]

\textbf{$\mu = \phi$:}

\[
0 = \frac{d^2\phi}{d\tau^2}
  + \Gamma^\phi_{\alpha\beta}\frac{dx^\alpha}{d\tau}\frac{dx^\beta}{d\tau}
\]

The non-zero pieces are $\Gamma^\phi_{r\phi}$ and $\Gamma^\phi_{\theta\phi}$:
\[
0 = \frac{d^2\phi}{d\tau^2}
  + 2\Gamma^\phi_{r\phi}\frac{dr}{d\tau}\frac{d\phi}{d\tau}
  + \Gamma^\phi_{\theta\phi}\frac{d\phi}{d\tau}\frac{d\theta}{d\tau}
\]

Using
\(
\Gamma^\phi_{r\phi} = \frac{r}{b^2 + r^2}
\)
and
\(
\Gamma^\phi_{\theta\phi} = \cot\theta
\),
then setting $\theta = \pi/2$ so that $d\theta/d\tau = 0$ and $\cot\theta = 0$,
\[
\frac{d^2\phi}{d\tau^2}
  = -\,\frac{2r}{b^2 + r^2}\frac{dr}{d\tau}\frac{d\phi}{d\tau}
\]

\examnote{These are the working geodesic equations used for the effective potential in part (d).}
\end{solutionbox}

% ---------- (d) ----------
\subsection*{(d) Circular orbits at the throat}

\begin{problembox}
The Schwarzschild metric admits geodesics that are circular orbits in a single plane around the mass $M$.  Is it possible for a free particle to orbit the throat of the wormhole in the same way?  Explain.
\end{problembox}

\begin{solutionbox}
We can use the effective potential method of Ch.\ 10.

First, note $2$ constants of integration from the $t$ and $\phi$ geodesics.

From the $t$ geodesic:
\[
0 = \frac{d}{d\tau}\!\left(g_{tt}\frac{dt}{d\tau}\right)
   \quad\Rightarrow\quad
   -\frac{dt}{d\tau} = \text{const} \equiv -e
\]
or
\[
\frac{dt}{d\tau} = e
\]

From the $\phi$ geodesic:
\[
0 = \frac{d}{d\tau}\!\left(g_{\phi\phi}\frac{d\phi}{d\tau}\right)
   = \frac{d}{d\tau}\!\left[(r^2 + b^2)\sin^2\theta\frac{d\phi}{d\tau}\right]
   \equiv \ell
\]
For $\theta = \pi/2$ this gives
\[
\ell = (r^2 + b^2)\frac{d\phi}{d\tau}
\]

Use $-1 = \vec{u}\cdot\vec{u}$ as in Box 10.2:
\[
-1 = g_{tt}\left(\frac{dt}{d\tau}\right)^2
      + g_{rr}\left(\frac{dr}{d\tau}\right)^2
      + g_{\theta\theta}\left(\frac{d\theta}{d\tau}\right)^2
      + g_{\phi\phi}\left(\frac{d\phi}{d\tau}\right)^2
\]

With $\theta = \pi/2$ and $d\theta/d\tau = 0$,
\[
-1 = -e^2 + \left(\frac{dr}{d\tau}\right)^2
      + (r^2 + b^2)\left(\frac{d\phi}{d\tau}\right)^2
\]

Using $\ell = (r^2 + b^2)\,d\phi/d\tau$,
\[
-1 = -e^2 + \left(\frac{dr}{d\tau}\right)^2
      + (r^2 + b^2)\left(\frac{\ell}{r^2 + b^2}\right)^2
\]

Or, as in (10.8),
\[
\underbrace{e^2 - 1}_{E}
 = \underbrace{\left(\frac{dr}{d\tau}\right)^2}_{K}
   + \underbrace{\frac{\ell^2}{r^2 + b^2}}_{V(r)}
\]

$V(r)$ has a maximum at $r = 0$, so an unstable circular orbit exists there.

\examnote{Key fact: $V(r)$ peaks at the throat, so circular orbits about the throat are possible but unstable.}
\end{solutionbox}

% ---------- (e) ----------
\subsection*{(e) Riemann tensor components}

\begin{problembox}
Show that all $R^t{}_{\beta\mu\nu} = 0$ and calculate one of $R^r{}_{\theta r\theta}$ or $R^r{}_{\phi r\phi}$.
\end{problembox}

\begin{solutionbox}
Start from
\[
R^\alpha{}_{\beta\rho\nu}
 = \partial_\rho\Gamma^\alpha_{\beta\nu}
   - \partial_\nu\Gamma^\alpha_{\beta\rho}
   + \Gamma^\alpha_{\rho\gamma}\Gamma^\gamma_{\beta\nu}
   - \Gamma^\alpha_{\nu\gamma}\Gamma^\gamma_{\beta\rho}
\]

All $\Gamma^t_{\mu\nu} = 0$, so
\[
R^t{}_{\beta\mu\nu} = 0
\]

\textbf{Compute $R^r{}_{\theta r\theta}$.}

\[
R^r{}_{\theta r\theta}
 = \partial_r\Gamma^r_{\theta\theta}
   - \partial_\theta\Gamma^r_{\theta r}
   + \Gamma^r_{r\gamma}\Gamma^\gamma_{\theta\theta}
   - \Gamma^r_{\theta\gamma}\Gamma^\gamma_{r\theta}
\]

Most terms vanish:
\[
\Gamma^r_{\theta r} = 0, \qquad
\Gamma^r_{rr} = 0, \qquad
\Gamma^\gamma_{\theta\theta} = 0\ \text{for all }\gamma
\]

Using $\Gamma^r_{\theta\theta} = -r$ and $\Gamma^\theta_{r\theta}
 = \Gamma^\theta_{\theta r} = r/(b^2 + r^2)$,
\begin{align*}
R^r{}_{\theta r\theta}
 &= \partial_r(-r) - 0 + 0
    - \Gamma^r_{\theta\theta}\Gamma^\theta_{r\theta} \\
 &= (-1) - \bigl(-r\bigr)\frac{r}{b^2 + r^2} \\
 &= -1 + \frac{r^2}{b^2 + r^2} \\
 &= -\,\frac{b^2}{b^2 + r^2}
\end{align*}

\textbf{Compute $R^r{}_{\phi r\phi}$.}

\[
R^r{}_{\phi r\phi}
 = \partial_r\Gamma^r_{\phi\phi}
   - \partial_\phi\Gamma^r_{\phi r}
   + \Gamma^r_{r\gamma}\Gamma^\gamma_{\phi\phi}
   - \Gamma^r_{\phi\gamma}\Gamma^\gamma_{r\phi}
\]

Again, many terms vanish.  Using $\Gamma^r_{\phi\phi} = -r\sin^2\theta$ and
$\Gamma^\phi_{r\phi} = \Gamma^\phi_{\phi r} = r/(b^2 + r^2)$,
\begin{align*}
R^r{}_{\phi r\phi}
 &= \partial_r(-r\sin^2\theta) - 0 + 0
    - \Gamma^r_{\phi\phi}\Gamma^\phi_{r\phi} \\
 &= -\sin^2\theta
    - \bigl(-r\sin^2\theta\bigr)\frac{r}{b^2 + r^2} \\
 &= -\sin^2\theta + \frac{r^2\sin^2\theta}{b^2 + r^2} \\
 &= -\,\frac{b^2\sin^2\theta}{b^2 + r^2}
\end{align*}

\examnote{$R^r{}_{\theta r\theta}$ and $R^r{}_{\phi r\phi}$ are the only independent non-zero components with an upper $r$ index that you need later for tidal forces.}
\end{solutionbox}

% ---------- (f) ----------
\subsection*{(f) Tidal forces and geodesic deviation}

\begin{problembox}
A person falling into a black hole will eventually be torn apart by tidal forces.  Will this happen to a person traversing a Morris--Thorne wormhole?
\end{problembox}

\begin{solutionbox}
Equation of geodesic deviation:
\[
\left(\frac{d^2 n}{d\tau^2}\right)^\alpha
 = - R^\alpha{}_{\beta\mu\nu} U^\beta n^\mu U^\nu
\]

Consider deviation along $r$:
\[
\left(\frac{d^2 n}{d\tau^2}\right)^r
 = - R^r{}_{\beta\mu\nu} U^\beta n^\mu U^\nu
\]

Since all $R^t{}_{\beta\mu\nu} = 0$ this reduces to
\[
\left(\frac{d^2 n}{d\tau^2}\right)^r
 = - R^r{}_{\theta r\theta}\,n^r (U^\theta)^2
   - R^r{}_{\phi r\phi}\,(U^\phi)^2
\]

Using the results from part (e),
\[
R^r{}_{\theta r\theta}
 = -\,\frac{b^2}{b^2 + r^2}, \qquad
R^r{}_{\phi r\phi}
 = -\,\frac{b^2\sin^2\theta}{b^2 + r^2}
\]
both finite at $r = 0$.

Thus, provided the person is not travelling too quickly, the tidal forces will be small.
\examnote{Key point: no divergent curvature at the throat, so no ``spaghettification'' like in a Schwarzschild singularity.}
\end{solutionbox}

% ---------- (g) ----------
\subsection*{(g) Expression for $R_{rr}$}

\begin{problembox}
Write down the expression for the $R_{rr}$ component of the Ricci tensor in terms of the non-zero Christoffel symbols and their derivatives, i.e.\ $R_{rr} = R^\alpha{}_{r\alpha r} = \ldots$.
\end{problembox}

\begin{solutionbox}
Start from
\[
R_{rr} = R^\alpha{}_{r\alpha r}
 = \partial_\alpha\Gamma^\alpha_{rr}
   - \partial_r\Gamma^\alpha_{r\alpha}
   + \Gamma^\alpha_{\alpha\gamma}\Gamma^\gamma_{rr}
   - \Gamma^\alpha_{r\gamma}\Gamma^\gamma_{r\alpha}
\]

Because $\Gamma^\alpha_{rr} = 0$ and $\Gamma^\gamma_{rr} = 0$,
\[
R_{rr}
 = - \partial_r\Gamma^\alpha_{r\alpha}
   - \Gamma^\alpha_{r\gamma}\Gamma^\gamma_{r\alpha}
\]

Only the angular components contribute:
\[
\Gamma^\theta_{r\theta} = \Gamma^\phi_{r\phi}
 = \frac{r}{b^2 + r^2}
\]

So
\[
R_{rr}
 = -\partial_r\Gamma^\theta_{r\theta}
   - \partial_r\Gamma^\phi_{r\phi}
   - \Gamma^\theta_{r\theta}\Gamma^\theta_{r\theta}
   - \Gamma^\phi_{r\phi}\Gamma^\phi_{r\phi}
\]

Using $\Gamma^\theta_{r\theta} = \Gamma^\phi_{r\phi}$,
\[
R_{rr}
 = -2\left[\partial_r\Gamma^\theta_{r\theta}
          + \left(\Gamma^\theta_{r\theta}\right)^2\right]
\]

\examnote{This matches the formula from the Diagonal Metric Worksheet for a metric with two identical angular directions.}
\end{solutionbox}

% ---------- (h) ----------
\subsection*{(h) Evaluating $R_{rr}$}

\begin{problembox}
Calculate $R_{rr}$ using either the expression above or the Diagonal Metric Worksheet.
\end{problembox}

\begin{solutionbox}
From part (g),
\[
R_{rr}
 = -2\left[\partial_r\Gamma^\theta_{r\theta}
          + \left(\Gamma^\theta_{r\theta}\right)^2\right]
\]
with
\[
\Gamma^\theta_{r\theta}
 = \frac{r}{b^2 + r^2}
\]

Compute the derivative:
\begin{align*}
\partial_r\Gamma^\theta_{r\theta}
 &= \partial_r\left(\frac{r}{b^2 + r^2}\right) \\
 &= \frac{(b^2 + r^2) - 2r^2}{(b^2 + r^2)^2} \\
 &= \frac{b^2 - r^2}{(b^2 + r^2)^2}
\end{align*}

Also
\[
\left(\Gamma^\theta_{r\theta}\right)^2
 = \left(\frac{r}{b^2 + r^2}\right)^2
 = \frac{r^2}{(b^2 + r^2)^2}
\]

Therefore
\begin{align*}
R_{rr}
 &= -2\left[\frac{b^2 - r^2}{(b^2 + r^2)^2}
            + \frac{r^2}{(b^2 + r^2)^2}\right] \\
 &= -2\,\frac{b^2}{(b^2 + r^2)^2}
\end{align*}

So
\[
R_{rr} = -\,\frac{2 b^2}{(b^2 + r^2)^2}
\]

\examnote{Only $R_{rr}$ is non-zero; the other diagonal components of $R_{\mu\nu}$ vanish for this metric.}
\end{solutionbox}

% Part (j)
\subsection*{(j) Stress-energy tensor and energy conditions}

\begin{solutionbox}
\textbf{Problem (j).} Calculate the components of the stress-energy tensor $T^{\mu\nu}$. Show that ``such material would violate all the `energy conditions' that underlie some deeply cherished theorems in general relativity (Morris \& Thorne 1988),''
or at any rate, that it violates the dominant energy condition.

\medskip

Start from the Einstein equation written in mixed-index form
\begin{align*}
T^{\mu\nu}
    &= \frac{1}{8\pi G}\Big[\,R^{\mu\nu} - \tfrac12 g^{\mu\nu} R\,\Big]
\end{align*}
and use the fact that
\begin{align*}
R^{\mu\nu}
    &= g^{\mu\alpha} g^{\nu\beta} R_{\alpha\beta}
    \qquad\text{(with $R_{\alpha\beta}$ non-zero only for $\alpha=\beta=r$)} \\
R^{rr}
    &= \big(g^{rr}\big)^2 R_{rr}
\end{align*}
From part (h), 
\[
R_{rr} = -\frac{2 b^2}{(r^2 + b^2)^2}
\]
and for the Morris--Thorne metric
\[
g^{tt} = -1, \qquad
g^{rr} = 1, \qquad
g^{\theta\theta} = \frac{1}{b^2 + r^2}, \qquad
g^{\phi\phi} = \frac{1}{(b^2 + r^2)\sin^2\theta}
\]

Since $g^{rr}=1$,
\[
R^{rr} = -\frac{2 b^2}{(r^2 + b^2)^2}
\]
and the Ricci scalar is
\begin{align*}
R &= R^{\mu\nu} R_{\mu\nu}
    = -\frac{2 b^2}{(r^2 + b^2)^2}
\end{align*}
\examnote{Here $R$ numerically equals $R_{rr}$ because $R_{\mu\nu}$ is non-zero only for $\mu=\nu=r$ and $g^{rr}=1$.}

\bigskip

\textbf{Component $T^{tt}$}

\begin{align*}
T^{tt}
    &= \frac{1}{8\pi G}\Big[\,R^{tt} - \tfrac12 g^{tt} R\,\Big] \\
    &= \frac{1}{8\pi G}\Big[\,0 - \tfrac12 g^{tt} R\,\Big] \\
    &= \frac{1}{8\pi G}\left[ -\tfrac12 (-1)\left(-\frac{2 b^2}{(r^2 + b^2)^2}\right)\right] \\
    &= \frac{1}{8\pi G}\left[ -\frac{b^2}{(r^2 + b^2)^2} \right]
\end{align*}

\textbf{Component $T^{rr}$}

\begin{align*}
T^{rr}
    &= \frac{1}{8\pi G}\Big[\,R^{rr} - \tfrac12 g^{rr} R\,\Big] \\
    &= \frac{1}{8\pi G}\left[ -\frac{2 b^2}{(r^2 + b^2)^2}
        - \tfrac12 (1)\left(-\frac{2 b^2}{(r^2 + b^2)^2}\right) \right] \\
    &= \frac{1}{8\pi G}\left[ -\frac{2 b^2}{(r^2 + b^2)^2}
        + \frac{b^2}{(r^2 + b^2)^2} \right] \\
    &= \frac{1}{8\pi G}\left[ -\frac{b^2}{(r^2 + b^2)^2} \right]
\end{align*}

\textbf{Component $T^{\theta\theta}$}

\begin{align*}
T^{\theta\theta}
    &= \frac{1}{8\pi G}\Big[\,R^{\theta\theta} - \tfrac12 g^{\theta\theta} R\,\Big] \\
    &= \frac{1}{8\pi G}\Big[\,0 - \tfrac12 g^{\theta\theta} R\,\Big] \\
    &= \frac{1}{8\pi G}\left[
        -\frac{1}{2(b^2 + r^2)}
        \left(-\frac{2 b^2}{(r^2 + b^2)^2}\right)
        \right] \\
    &= \frac{1}{8\pi G}\left[
        \frac{b^2}{(b^2 + r^2)^3}
        \right]
\end{align*}

\textbf{Component $T^{\phi\phi}$}

\begin{align*}
T^{\phi\phi}
    &= \frac{1}{8\pi G}\Big[\,R^{\phi\phi} - \tfrac12 g^{\phi\phi} R\,\Big] \\
    &= \frac{1}{8\pi G}\Big[\,0 - \tfrac12 g^{\phi\phi} R\,\Big] \\
    &= \frac{1}{8\pi G}\left[
        -\frac{1}{2(b^2 + r^2)\sin^2\theta}
        \left(-\frac{2 b^2}{(r^2 + b^2)^2}\right)
        \right] \\
    &= \frac{1}{8\pi G}\left[
        \frac{b^2}{(b^2 + r^2)^3\sin^2\theta}
        \right]
\end{align*}

So the non-zero components are
\begin{align*}
T^{tt} &= \frac{1}{8\pi G}\left[ -\frac{b^2}{(r^2 + b^2)^2} \right] \\
T^{rr} &= \frac{1}{8\pi G}\left[ -\frac{b^2}{(r^2 + b^2)^2} \right] \\
T^{\theta\theta} &= \frac{1}{8\pi G}\left[
        \frac{b^2}{(b^2 + r^2)^3}
        \right] \\
T^{\phi\phi} &= \frac{1}{8\pi G}\left[
        \frac{b^2}{(b^2 + r^2)^3\sin^2\theta}
        \right]
\end{align*}

\examnote{In an orthonormal frame, $T^{tt}$ corresponds to $-\rho$. Since $T^{tt}<0$, the energy density $\rho$ is negative.}

\medskip

\textbf{Energy-condition statement}

From $T^{tt}<0$ we see the \emph{weak energy condition} (which requires $\rho>0$) fails:
\[
\text{weak energy condition:}\quad \rho>0 \ \text{fails}\quad (\rho<0)
\]
The \emph{dominant energy condition} (DEC), which requires $\rho\geq |p|$, also fails:
\[
\text{DEC:}\quad \rho\geq |p| \ \text{likewise fails}
\]
Thus the matter required to support the Morris--Thorne wormhole violates the usual energy conditions.

\end{solutionbox}


\end{document}
