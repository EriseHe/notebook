\documentclass[11pt,a4paper]{article}

% --- Essential Packages ---
\usepackage{amsmath, amssymb, amsthm} % Math symbols and environments
\usepackage{physics}      % Physics notation (e.g., \bra, \ket, \pdv)
\usepackage{geometry}     % Page layout
\usepackage{fancyhdr}     % Custom headers/footers
\usepackage{cancel}       % Crossing out terms
\usepackage{xcolor}       % Color support
\usepackage[many]{tcolorbox} % Colorful boxes for solutions
\usepackage{enumitem}     % Custom lists
\usepackage{tikz}         % Drawing diagrams
\usepackage{graphicx}     % Inserting images

% --- Page Layout ---
\geometry{top=0.7in, bottom=0.7in, left=0.7in, right=0.7in}
\setlength{\parindent}{0pt}
\setlength{\parskip}{0.5em}

% --- Custom Colors ---
\definecolor{graytext}{gray}{0.5}

% --- Custom Commands ---
\newcommand{\gray}[1]{{\color{graytext}#1}}
\newcommand{\examnote}[1]{{\color{red}\footnotesize\textbf{[Exam Note: #1]}}}

% --- Header Configuration ---
\newcommand{\courseName}{PHYS 458}
\newcommand{\hwNumber}{HW 5} 
\newcommand{\hwTopic}{Schwarzschild, Riemann \& Tides}

\pagestyle{fancy}
\fancyhf{}
\lhead{\textbf{\courseName: \hwNumber}}
\rhead{\hwTopic}
\cfoot{\thepage}

% --- Custom Solution Box Environment ---
\newtcolorbox{solutionbox}{
    breakable,
    enhanced,
    colback=white,
    colframe=black,
    title=\textbf{Derivation Transcript},
    fonttitle=\bfseries\small,
    arc=0mm,
    boxrule=1pt,
    left=5mm, right=5mm, top=5mm, bottom=5mm,
    before skip=10pt,
    after skip=20pt
}

\newtcolorbox{problembox}{
    colback=gray!10!white,
    colframe=gray!60!black,
    title=\textbf{Problem Statement},
    fonttitle=\bfseries\small,
    arc=0mm,
    boxrule=1pt,
    left=5mm, right=5mm, top=5mm, bottom=5mm
}

\title{\vspace{-2cm} \courseName: \hwNumber \ Solutions}
\author{}
\date{}

\begin{document}

%========================================
\section*{1. Christoffel Symbols for Schwarzschild}

\begin{problembox}
Use the method of Box 17.6 to calculate Schwarzschild Christoffel symbols
that were not calculated in that box (for example, the symbols
$\Gamma^{r}{}_{\mu\nu}$ and $\Gamma^{\phi}{}_{\mu\nu}$).
Compare the two geodesic equations to read off the Christoffel symbols.
\end{problembox}

\subsection*{(a) Symbols with upper index $\mu = \phi$}

\begin{solutionbox}

The two forms of the geodesic equation that we want to compare are

\begin{align*}
\text{(I)}\quad
0 &= \dv{\tau}\!\left(g_{\mu\nu}\dv{x^{\nu}}{\tau}\right)
      - \frac12 \partial_{\mu}\!
      \left(g_{\alpha\beta}\dv{x^{\alpha}}{\tau}\dv{x^{\beta}}{\tau}\right) \\
\text{(II)}\quad
0 &= \dv[2]{x^{\mu}}{\tau}
   + \Gamma^{\mu}{}_{\alpha\beta}
     \dv{x^{\alpha}}{\tau}\dv{x^{\beta}}{\tau} \, .
\end{align*}

For Schwarzschild in standard coordinates
$(t,r,\theta,\phi)$ we have
$g_{\phi\phi} = r^{2}\sin^{2}\theta$ and $g_{\mu\nu}$ is
independent of $\phi$.

We now take $\mu=\phi$ in (I):

\[
\boxed{\mu = \phi}
\]

\begin{align*}
\text{(I)}\quad
0
&= \dv{\tau}\!\left(g_{\phi\phi}\dv{\phi}{\tau}\right)
   - \frac12 \partial_{\phi}
   \left( g_{\alpha\beta}
          \dv{x^{\alpha}}{\tau}\dv{x^{\beta}}{\tau}
   \right)
\\
&= \dv{\tau}\!\left(g_{\phi\phi}\dv{\phi}{\tau}\right)
   - \frac12 \underbrace{\partial_{\phi}
   \left( g_{\alpha\beta}
          \dv{x^{\alpha}}{\tau}\dv{x^{\beta}}{\tau}
   \right)}_{\displaystyle 0\ \text{(no $\phi$-dependence)}} \, .
\end{align*}

So we keep only the first term:

\begin{align*}
0 &= \dv{\tau}\!\left(g_{\phi\phi}\dv{\phi}{\tau}\right) \\
  &= g_{\phi\phi}\dv[2]{\phi}{\tau}
     + \partial_{\theta}g_{\phi\phi}\dv{\theta}{\tau}\dv{\phi}{\tau}
     + \partial_{r}g_{\phi\phi}\dv{r}{\tau}\dv{\phi}{\tau} \, .
\end{align*}

Now use $g_{\phi\phi}=r^{2}\sin^{2}\theta$:

\begin{align*}
0
&= r^{2}\sin^{2}\theta \dv[2]{\phi}{\tau}
   + 2r^{2}\sin\theta\cos\theta\,
       \dv{\theta}{\tau}\dv{\phi}{\tau}
   + 2r\sin^{2}\theta\,
       \dv{r}{\tau}\dv{\phi}{\tau} \\
&\underbrace{=}_{\displaystyle
\text{divide by }r^{2}\sin^{2}\theta}
\dv[2]{\phi}{\tau}
 + 2\cot\theta\,
   \dv{\theta}{\tau}\dv{\phi}{\tau}
 + \frac{2}{r}\,
   \dv{r}{\tau}\dv{\phi}{\tau} \, .
\end{align*}

Now write the geodesic with $\mu=\phi$ in the form (II):

\begin{align*}
\text{(II)}\quad
0 &= \dv[2]{\phi}{\tau}
 + \Gamma^{\phi}{}_{\theta\phi}
   \dv{\theta}{\tau}\dv{\phi}{\tau}
 + \Gamma^{\phi}{}_{r\phi}
   \dv{r}{\tau}\dv{\phi}{\tau} \, .
\end{align*}

Comparing term by term, we identify

\begin{align*}
\Gamma^{\phi}{}_{\theta\phi} &= \cot\theta \, ,\\
\Gamma^{\phi}{}_{r\phi}      &= \frac{1}{r} \, ,
\end{align*}

and all other $\Gamma^{\phi}{}_{\mu\nu}$ vanish:
\[
\Gamma^{\phi}{}_{\mu\nu} = 0
\quad\text{for all other index combinations.}
\]

\examnote{The trick here is: use form (I), expand, and then
compare to form (II) to read off the Christoffels.}

\end{solutionbox}

\subsection*{(b) Symbols with upper index $\mu = r$}

\begin{solutionbox}

Again start from the pair of geodesic equations:

\begin{align*}
\text{(I)}\quad
0 &= \dv{\tau}\!\left(g_{\mu\nu}\dv{x^{\nu}}{\tau}\right)
      - \frac12 \partial_{\mu}\!
      \left(g_{\alpha\beta}\dv{x^{\alpha}}{\tau}\dv{x^{\beta}}{\tau}\right) \\
\text{(II)}\quad
0 &= \dv[2]{x^{\mu}}{\tau}
   + \Gamma^{\mu}{}_{\alpha\beta}
     \dv{x^{\alpha}}{\tau}\dv{x^{\beta}}{\tau} \, .
\end{align*}

Now take $\mu = r$.

\[
\boxed{\mu = r}
\]

\textbf{Step I: write geodesic in form (I).}

\begin{align*}
0
&= g_{rr}\dv[2]{r}{\tau}
   + \partial_{r}g_{rr}\left(\dv{r}{\tau}\right)^{2}
   - \frac12\partial_{r}\Big[
        g_{tt}\left(\dv{t}{\tau}\right)^{2}
      + g_{rr}\left(\dv{r}{\tau}\right)^{2}
      + g_{\theta\theta}\left(\dv{\theta}{\tau}\right)^{2}
      + g_{\phi\phi}\left(\dv{\phi}{\tau}\right)^{2}
     \Big] \, .
\end{align*}

For Schwarzschild,
\begin{align*}
g_{tt} &= -\left(1-\frac{2GM}{r}\right) ,\\
g_{rr} &= \left(1-\frac{2GM}{r}\right)^{-1} ,\\
g_{\theta\theta} &= r^{2} ,\\
g_{\phi\phi} &= r^{2}\sin^{2}\theta \, .
\end{align*}

The $r$-derivatives are
\begin{align*}
\partial_{r}g_{rr}
   &= -\left(1-\frac{2GM}{r}\right)^{-2}
      \left(\frac{2GM}{r^{2}}\right) , \\
\partial_{r}g_{tt}
   &= -\frac{2GM}{r^{2}} ,\\
\partial_{r}g_{\theta\theta}
   &= 2r ,\\
\partial_{r}g_{\phi\phi}
   &= 2r\sin^{2}\theta \, .
\end{align*}

\textbf{Step II: substitute these into (I).}

\begin{align*}
0
&= \left(1-\frac{2GM}{r}\right)^{-1}\dv[2]{r}{\tau}
   -\left(1-\frac{2GM}{r}\right)^{-2}
      \left(\frac{2GM}{r^{2}}\right)
      \left(\dv{r}{\tau}\right)^{2} \\
&\quad -\frac12 \Bigg[
   -\frac{2GM}{r^{2}}
   \left(\dv{t}{\tau}\right)^{2}
   -\left(1-\frac{2GM}{r}\right)^{-2}
     \left(\frac{2GM}{r^{2}}\right)
     \left(\dv{r}{\tau}\right)^{2} \\
&\qquad\qquad
   + 2r\left(\dv{\theta}{\tau}\right)^{2}
   + 2r\sin^{2}\theta\left(\dv{\phi}{\tau}\right)^{2}
   \Bigg] \, .
\end{align*}

Distribute the $-\frac12$ in the last line:

\begin{align*}
0
&= \left(1-\frac{2GM}{r}\right)^{-1}\dv[2]{r}{\tau}
   -\left(1-\frac{2GM}{r}\right)^{-2}
      \left(\frac{2GM}{r^{2}}\right)
      \left(\dv{r}{\tau}\right)^{2} \\
&\quad + \frac{GM}{r^{2}}
   \left(\dv{t}{\tau}\right)^{2}
   + \frac12\left(1-\frac{2GM}{r}\right)^{-2}
      \left(\frac{2GM}{r^{2}}\right)
      \left(\dv{r}{\tau}\right)^{2} \\
&\quad - r\left(\dv{\theta}{\tau}\right)^{2}
      - r\sin^{2}\theta\left(\dv{\phi}{\tau}\right)^{2} \, .
\end{align*}

Now factor where convenient.
First multiply the entire equation by
$\left(1-\dfrac{2GM}{r}\right)$ to isolate
$\dv[2]{r}{\tau}$:

\begin{align*}
0
&= \dv[2]{r}{\tau}
   -\left(1-\frac{2GM}{r}\right)^{-1}
      \left(\frac{2GM}{r^{2}}\right)
      \left(\dv{r}{\tau}\right)^{2} \\
&\quad + \frac{GM}{r^{2}}
   \left(1-\frac{2GM}{r}\right)
   \left(\dv{t}{\tau}\right)^{2} \\
&\quad - r\left(1-\frac{2GM}{r}\right)
      \left(\dv{\theta}{\tau}\right)^{2}
      - r\sin^{2}\theta\left(1-\frac{2GM}{r}\right)
        \left(\dv{\phi}{\tau}\right)^{2} \, .
\end{align*}

So we have

\begin{align*}
0
&= \dv[2]{r}{\tau}
   +\frac{GM}{r^{2}}
     \left(1-\frac{2GM}{r}\right)
     \left(\dv{t}{\tau}\right)^{2}
   -\frac{GM}{r^{2}}
     \left(1-\frac{2GM}{r}\right)^{-1}
     \left(\dv{r}{\tau}\right)^{2} \\
&\quad - r\left(1-\frac{2GM}{r}\right)
          \left(\dv{\theta}{\tau}\right)^{2}
      - r\sin^{2}\theta\left(1-\frac{2GM}{r}\right)
        \left(\dv{\phi}{\tau}\right)^{2} \, .
\end{align*}

\textbf{Step III: compare with the geodesic in form (II).}

For $\mu = r$, equation (II) is

\[
0 = \dv[2]{r}{\tau}
    + \Gamma^{r}{}_{tt}\left(\dv{t}{\tau}\right)^{2}
    + \Gamma^{r}{}_{rr}\left(\dv{r}{\tau}\right)^{2}
    + \Gamma^{r}{}_{\theta\theta}\left(\dv{\theta}{\tau}\right)^{2}
    + \Gamma^{r}{}_{\phi\phi}\left(\dv{\phi}{\tau}\right)^{2} \, .
\]

Comparing coefficient by coefficient, we read off

\begin{align*}
\Gamma^{r}{}_{tt}
 &= \frac{GM}{r^{2}}\left(1-\frac{2GM}{r}\right) ,\\[4pt]
\Gamma^{r}{}_{rr}
 &= -\frac{GM}{r^{2}}\left(1-\frac{2GM}{r}\right)^{-1} ,\\[4pt]
\Gamma^{r}{}_{\theta\theta}
 &= -r\left(1-\frac{2GM}{r}\right) ,\\[4pt]
\Gamma^{r}{}_{\phi\phi}
 &= -r\sin^{2}\theta\left(1-\frac{2GM}{r}\right) ,
\end{align*}

and all other $\Gamma^{r}{}_{\mu\nu}$ vanish:
\[
\Gamma^{r}{}_{\mu\nu}=0 \quad\text{for all other index combinations.}
\]

\examnote{These are the standard Schwarzschild $\Gamma^{r}{}_{\mu\nu}$.
If you remember the final forms above, you can very quickly build any
radial geodesic equation on the exam.}

\end{solutionbox}

%========================================
\section*{2. Riemann Tensor from Commutator of Covariant Derivatives}

\begin{problembox}
Show that the commutator of the covariant derivatives of a vector
gives an expression for the Riemann tensor:
\[
\nabla_{\mu}\nabla_{\nu}V^{\rho}
 - \nabla_{\nu}\nabla_{\mu}V^{\rho}
 = R^{\rho}{}_{\sigma\mu\nu}V^{\sigma} \, .
\]
\end{problembox}

\begin{solutionbox}

Start from the definition of the covariant derivative of a vector:
\[
\nabla_{\mu}V^{\rho}
 = \partial_{\mu}V^{\rho}
 + \Gamma^{\rho}{}_{\mu\sigma}V^{\sigma} \, .
\]

First compute $\nabla_{\nu}V^{\rho}$, then $\nabla_{\mu}$ of that:

\begin{align*}
\nabla_{\mu}\nabla_{\nu}V^{\rho}
&= \nabla_{\mu}
   \left(\partial_{\nu}V^{\rho}
        + \Gamma^{\rho}{}_{\nu\lambda}V^{\lambda}\right) \\[4pt]
&= \partial_{\mu}\left(\partial_{\nu}V^{\rho}
        + \Gamma^{\rho}{}_{\nu\lambda}V^{\lambda}\right)
 + \Gamma^{\rho}{}_{\mu\sigma}
   \left(\partial_{\nu}V^{\sigma}
        + \Gamma^{\sigma}{}_{\nu\lambda}V^{\lambda}\right)
 - \Gamma^{\lambda}{}_{\mu\nu}
   \left(\partial_{\lambda}V^{\rho}
        + \Gamma^{\rho}{}_{\lambda\sigma}V^{\sigma}\right) \, .
\end{align*}

Expand everything:

\begin{align*}
\nabla_{\mu}\nabla_{\nu}V^{\rho}
&= \partial_{\mu}\partial_{\nu}V^{\rho}
 + (\partial_{\mu}\Gamma^{\rho}{}_{\nu\lambda})V^{\lambda}
 + \Gamma^{\rho}{}_{\nu\lambda}\partial_{\mu}V^{\lambda} \\
&\quad
 + \Gamma^{\rho}{}_{\mu\sigma}\partial_{\nu}V^{\sigma}
 + \Gamma^{\rho}{}_{\mu\sigma}\Gamma^{\sigma}{}_{\nu\lambda}V^{\lambda} \\
&\quad
 - \Gamma^{\lambda}{}_{\mu\nu}\partial_{\lambda}V^{\rho}
 - \Gamma^{\lambda}{}_{\mu\nu}\Gamma^{\rho}{}_{\lambda\sigma}V^{\sigma} \, .
\end{align*}

Similarly, exchange $\mu\leftrightarrow\nu$ to get
$\nabla_{\nu}\nabla_{\mu}V^{\rho}$:

\begin{align*}
\nabla_{\nu}\nabla_{\mu}V^{\rho}
&= \partial_{\nu}\partial_{\mu}V^{\rho}
 + (\partial_{\nu}\Gamma^{\rho}{}_{\mu\lambda})V^{\lambda}
 + \Gamma^{\rho}{}_{\mu\lambda}\partial_{\nu}V^{\lambda} \\
&\quad
 + \Gamma^{\rho}{}_{\nu\sigma}\partial_{\mu}V^{\sigma}
 + \Gamma^{\rho}{}_{\nu\sigma}\Gamma^{\sigma}{}_{\mu\lambda}V^{\lambda} \\
&\quad
 - \Gamma^{\lambda}{}_{\nu\mu}\partial_{\lambda}V^{\rho}
 - \Gamma^{\lambda}{}_{\nu\mu}\Gamma^{\rho}{}_{\lambda\sigma}V^{\sigma} \, .
\end{align*}

Now subtract:
\begin{align*}
\nabla_{\mu}\nabla_{\nu}V^{\rho}
 - \nabla_{\nu}\nabla_{\mu}V^{\rho}&= \big[\partial_{\mu}\partial_{\nu}V^{\rho}
      - \partial_{\nu}\partial_{\mu}V^{\rho}\big] \\
&\quad
 + \big[(\partial_{\mu}\Gamma^{\rho}{}_{\nu\lambda})
        - (\partial_{\nu}\Gamma^{\rho}{}_{\mu\lambda})\big]V^{\lambda} \\
&\quad
 + \big[\Gamma^{\rho}{}_{\nu\lambda}\partial_{\mu}V^{\lambda}
        - \Gamma^{\rho}{}_{\mu\lambda}\partial_{\nu}V^{\lambda}\big] \\
&\quad
 + \big[\Gamma^{\rho}{}_{\mu\sigma}\partial_{\nu}V^{\sigma}
        - \Gamma^{\rho}{}_{\nu\sigma}\partial_{\mu}V^{\sigma}\big] \\
&\quad
 + \big[\Gamma^{\rho}{}_{\mu\sigma}\Gamma^{\sigma}{}_{\nu\lambda}
        - \Gamma^{\rho}{}_{\nu\sigma}\Gamma^{\sigma}{}_{\mu\lambda}\big]V^{\lambda} \\
&\quad
 - \big[\Gamma^{\lambda}{}_{\mu\nu}
        - \Gamma^{\lambda}{}_{\nu\mu}\big]\partial_{\lambda}V^{\rho} \\
&\quad
 - \big[\Gamma^{\lambda}{}_{\mu\nu}
        - \Gamma^{\lambda}{}_{\nu\mu}\big]\Gamma^{\rho}{}_{\lambda\sigma}V^{\sigma} \, .
\end{align*}

Now, using the symmetry $\Gamma^{\lambda}{}_{\mu\nu}
= \Gamma^{\lambda}{}_{\nu\mu}$, the terms with
$\partial_{\lambda}V^{\rho}$ and the last group proportional to
$\Gamma^{\rho}{}_{\lambda\sigma}V^{\sigma}$ cancel.  
Similarly, the mixed terms with $\partial_{\mu}V^{\lambda}$ and
$\partial_{\nu}V^{\lambda}$ cancel pairwise
(the professor circled these ``red'' terms on the sheet).

Thus we are left only with terms proportional to $V^{\lambda}$:

\begin{align*}
\nabla_{\mu}\nabla_{\nu}V^{\rho}
 - \nabla_{\nu}\nabla_{\mu}V^{\rho}
&=
\Big[
   \partial_{\mu}\Gamma^{\rho}{}_{\nu\lambda}
 - \partial_{\nu}\Gamma^{\rho}{}_{\mu\lambda}
 + \Gamma^{\rho}{}_{\mu\sigma}\Gamma^{\sigma}{}_{\nu\lambda}
 - \Gamma^{\rho}{}_{\nu\sigma}\Gamma^{\sigma}{}_{\mu\lambda}
\Big] V^{\lambda} \, .
\end{align*}

By definition this bracket is the Riemann tensor:

\[
R^{\rho}{}_{\lambda\mu\nu}
 = \partial_{\mu}\Gamma^{\rho}{}_{\nu\lambda}
 - \partial_{\nu}\Gamma^{\rho}{}_{\mu\lambda}
 + \Gamma^{\rho}{}_{\mu\sigma}\Gamma^{\sigma}{}_{\nu\lambda}
 - \Gamma^{\rho}{}_{\nu\sigma}\Gamma^{\sigma}{}_{\mu\lambda} \, .
\]

Renaming the dummy index $\lambda\to\sigma$ gives the final result

\[
\boxed{
\nabla_{\mu}\nabla_{\nu}V^{\rho}
 - \nabla_{\nu}\nabla_{\mu}V^{\rho}
 = R^{\rho}{}_{\sigma\mu\nu}V^{\sigma}
}
\]

exactly as required.

\examnote{On the exam, the key is: expand both covariant derivatives,
subtract, and remember (i) $\Gamma^{\lambda}{}_{\mu\nu}$ is symmetric
in the lower indices, and (ii) partial derivatives commute on scalars.}

\end{solutionbox}

%========================================
\section*{3. Computing $R^{t}{}_{r t r}$ in Schwarzschild}

\begin{problembox}
Show that
\[
R^{t}{}_{rtr}
 = \frac{2GM}{r^{3}}\left(1-\frac{2GM}{r}\right)^{-1} \, .
\]
Then discuss the behavior at $r=2GM$ and as $r\to\infty$.
\end{problembox}

\begin{solutionbox}

The relevant nonzero Christoffel symbols (from Problem 1 and Box 17.6) are
\begin{align*}
\Gamma^{t}{}_{tr} &= \Gamma^{t}{}_{rt}
  = \frac{GM}{r^{2}}\left(1-\frac{2GM}{r}\right)^{-1} ,\\[4pt]
\Gamma^{r}{}_{tt} &=
  \frac{GM}{r^{2}}\left(1-\frac{2GM}{r}\right) ,\\[4pt]
\Gamma^{r}{}_{rr} &=
  -\frac{GM}{r^{2}}\left(1-\frac{2GM}{r}\right)^{-1} .
\end{align*}

For the component $R^{t}{}_{rtr}$ we use the definition
\[
R^{\rho}{}_{\sigma\mu\nu}
 = \partial_{\mu}\Gamma^{\rho}{}_{\nu\sigma}
  - \partial_{\nu}\Gamma^{\rho}{}_{\mu\sigma}
  + \Gamma^{\rho}{}_{\mu\lambda}\Gamma^{\lambda}{}_{\nu\sigma}
  - \Gamma^{\rho}{}_{\nu\lambda}\Gamma^{\lambda}{}_{\mu\sigma} \, .
\]

Set $(\rho,\sigma,\mu,\nu)=(t,r,t,r)$:

\begin{align*}
R^{t}{}_{rtr}
 &= \partial_{t}\Gamma^{t}{}_{rr}
    - \partial_{r}\Gamma^{t}{}_{tr}
    + \Gamma^{t}{}_{t\lambda}\Gamma^{\lambda}{}_{rr}
    - \Gamma^{t}{}_{r\lambda}\Gamma^{\lambda}{}_{tr} \, .
\end{align*}

Schwarzschild is static, so $\partial_{t}\Gamma^{t}{}_{rr}=0$.

Only $\lambda=t,r$ contribute.  Thus

\begin{align*}
R^{t}{}_{rtr}
 &= - \partial_{r}\Gamma^{t}{}_{tr}
    + \Gamma^{t}{}_{tt}\Gamma^{t}{}_{rr}
    + \Gamma^{t}{}_{tr}\Gamma^{r}{}_{rr}
    - \Gamma^{t}{}_{rt}\Gamma^{t}{}_{tr}
    - \Gamma^{t}{}_{rr}\Gamma^{r}{}_{tr} \, .
\end{align*}

But $\Gamma^{t}{}_{tt}=0$ and $\Gamma^{t}{}_{rr}=0$ for Schwarzschild in
these coordinates, so the only surviving terms are

\begin{align*}
R^{t}{}_{rtr}
 &= -\partial_{r}\Gamma^{t}{}_{tr}
    + \Gamma^{t}{}_{tr}\Gamma^{r}{}_{rr} \, .
\end{align*}

Now compute the derivative:
\[
\Gamma^{t}{}_{tr}
 = \frac{GM}{r^{2}}\left(1-\frac{2GM}{r}\right)^{-1}
 = GM\,r^{-2}\left(1-\frac{2GM}{r}\right)^{-1} .
\]

\begin{align*}
\partial_{r}\Gamma^{t}{}_{tr}
&= GM\left[
   -2r^{-3}\left(1-\frac{2GM}{r}\right)^{-1}
   + r^{-2}\left(1-\frac{2GM}{r}\right)^{-2}
     \left(\frac{2GM}{r^{2}}\right) \right]  \\
&= -\frac{2GM}{r^{3}}
   \left(1-\frac{2GM}{r}\right)^{-1}
   + \frac{2G^{2}M^{2}}{r^{4}}
     \left(1-\frac{2GM}{r}\right)^{-2} \, .
\end{align*}

Next,
\[
\Gamma^{t}{}_{tr}\Gamma^{r}{}_{rr}
 = \left[\frac{GM}{r^{2}}
        \left(1-\frac{2GM}{r}\right)^{-1}\right]
   \left[-\frac{GM}{r^{2}}
        \left(1-\frac{2GM}{r}\right)^{-1}\right]
 = -\frac{G^{2}M^{2}}{r^{4}}
   \left(1-\frac{2GM}{r}\right)^{-2} \, .
\]

Therefore

\begin{align*}
R^{t}{}_{rtr}
&= -\partial_{r}\Gamma^{t}{}_{tr}
   + \Gamma^{t}{}_{tr}\Gamma^{r}{}_{rr} \\[4pt]
&= -\left[
      -\frac{2GM}{r^{3}}
        \left(1-\frac{2GM}{r}\right)^{-1}
      + \frac{2G^{2}M^{2}}{r^{4}}
        \left(1-\frac{2GM}{r}\right)^{-2}
    \right]
   -\frac{G^{2}M^{2}}{r^{4}}
      \left(1-\frac{2GM}{r}\right)^{-2} \\[4pt]
&= \frac{2GM}{r^{3}}
     \left(1-\frac{2GM}{r}\right)^{-1} \, .
\end{align*}

So we reproduce the desired result:

\[
\boxed{
R^{t}{}_{rtr}
 = \frac{2GM}{r^{3}}\left(1-\frac{2GM}{r}\right)^{-1}
}
\]

Now examine the limits:

\begin{itemize}
\item At $r = 2GM$,
\[
R^{t}{}_{rtr} \to \infty
\]
in these coordinates.
This divergence is a \emph{coordinate pathology} of Schwarzschild
coordinates, not a physical singularity.

\item As $r\to\infty$,
\[
R^{t}{}_{rtr} \to 0
\]
which reflects that the spacetime is asymptotically flat far from the
mass.
\end{itemize}

\examnote{The important pattern to remember:
curvature scales like $GM/r^{3}$.  Factors of
$\left(1-\dfrac{2GM}{r}\right)^{-1}$ are coordinate artifacts of
Schwarzschild coordinates.}

\end{solutionbox}

%========================================
\section*{4. Local-Orthonormal-Frame Curvature Component}

\begin{solutionbox}

On the professor's sheet this is labeled ``19.8: investigate nature of
singularity at $r=2GM$ in Schwarzschild metric.''

The main goal is to show that the physically measured component
$R_{\hat{t}\hat{r}\hat{t}\hat{r}}$ in a local orthonormal frame (LOF)
is finite at $r=2GM$ even though $R^{t}{}_{rtr}$ diverges in
Schwarzschild coordinates.

Let $\tilde{O}_{\mu}$ denote the orthonormal basis vectors of the
local frame.  For a 4--vector $A^{\mu}$, the components measured by the
observer are
\[
A^{\mu}_{\text{obs}} = \tilde{O}^{\mu}{}_{\rho} A^{\rho} \, .
\]

For a rank--$(1,3)$ tensor such as $R^{\alpha}{}_{\beta\mu\nu}$, its
components in the observer's orthonormal frame are

\[
(R^{\alpha}{}_{\beta\mu\nu})_{\text{obs}}
 = R^{\alpha}{}_{\beta\mu\nu}
   (\tilde{O}^{\beta})_{\hat{b}}
   (\tilde{O}^{\mu})_{\hat{m}}
   (\tilde{O}^{\nu})_{\hat{n}}
   (\tilde{O}_{\alpha})^{\hat{a}} \, ,
\]

with sums over repeated hatted indices.
By convention the $\hat{z}$-direction of the LOF is aligned with the
radial direction $r$, so the component of physical interest is
$R_{\hat{z}\hat{t}\hat{z}\hat{t}}$, which the sheet denotes
by $R^{+}_{z+z+}$ in the LOF.

Using the explicit tetrad from Problem 12.7 (quoted on the sheet),
\[
\tilde{O}^{\mu}{}_{\hat{z}}
 =
 \begin{pmatrix}
 -\dfrac{\sqrt{2GM/r}}{\left(1-\dfrac{2GM}{r}\right)} \\
 \left(1-\dfrac{2GM}{r}\right)^{-1} \\
 0 \\ 0
 \end{pmatrix},
\qquad
\tilde{O}^{\mu}{}_{\hat{t}}
 =
 \begin{pmatrix}
 \left(1-\dfrac{2GM}{r}\right)^{-1} \\
 -\sqrt{\dfrac{2GM}{r}}\left(1-\dfrac{2GM}{r}\right)^{-1} \\
 0 \\ 0
 \end{pmatrix},
\]
one finds after substituting into the transformation law and summing
over indices (as done line by line on the handwritten sheet) that

\[
(R^{+}_{z+z+})_{\text{obs}}
 = -g_{tt} R^{t}{}_{rtr} \, .
\]

Since $g_{tt} = -\left(1-\dfrac{2GM}{r}\right)$ and we already found
$R^{t}{}_{rtr}$ above, we get

\begin{align*}
(R^{+}_{z+z+})_{\text{obs}}
&= -g_{tt} R^{t}{}_{rtr} \\
&= \left(1-\frac{2GM}{r}\right)
   \left[
     \frac{2GM}{r^{3}}
     \left(1-\frac{2GM}{r}\right)^{-1}
   \right] \\
&= \boxed{\frac{2GM}{r^{3}}} \, .
\end{align*}

This is the curvature component as measured in the local inertial
frame of a freely falling observer.  It is perfectly \emph{finite} at
$r=2GM$ and only diverges as $r\to 0$.

\examnote{The key physical message: the Schwarzschild-coordinate
singularity at $r=2GM$ is \emph{not} a true curvature singularity.
Real tidal forces are determined by LOF components like
$R^{+}_{z+z+} = 2GM/r^{3}$.}

\end{solutionbox}

%========================================
\section*{5. Geodesic Deviation and Tidal Forces}

\begin{problembox}
Consider a star with radius $r_{\ast}$ falling radially into a black
hole of mass $M$.  Use the inertial frame of the star where it is
locally at rest and consider the tidal force in the radial ($+z$)
direction.  The four--velocity and separation vector are
\[
u^{\mu} =
\begin{bmatrix}
1 \\ 0 \\ 0 \\ 0
\end{bmatrix},
\qquad
n^{\mu} =
\begin{bmatrix}
0 \\ 0 \\ 0 \\ r_{\ast}
\end{bmatrix}.
\]
\begin{enumerate}[label=(\alph*)]
\item Find the tidal acceleration $\left(\dfrac{d^{2}n}{d\tau^{2}}\right)^{z}$.
\item Estimate the disruption radius $r_{\text{disrupt}}$ where the
tidal acceleration equals the surface gravity of the star.
\end{enumerate}
\end{problembox}

\begin{solutionbox}

The equation of geodesic deviation is
\[
\left(\dv[2]{n^{\alpha}}{\tau}\right)
 = - R^{\alpha}{}_{\beta\mu\nu}
     u^{\beta}u^{\mu}n^{\nu} \, .
\]

On the professor's sheet this is applied with the local orthonormal
frame where the freely falling frame (LIF) has

\[
u^{\mu} =
\begin{bmatrix}
1 \\ 0 \\ 0 \\ 0
\end{bmatrix},
\qquad
n^{\mu} =
\begin{bmatrix}
0 \\ 0 \\ 0 \\ r_{\ast}
\end{bmatrix},
\]

and the curvature component from the previous problem,
\[
(R^{+}_{z+z+})_{\text{obs}} = \frac{2GM}{r^{3}} \, .
\]

\subsubsection*{(a) Tidal acceleration in the $z$-direction}

Take $\alpha = z$ and note that only $\beta=\mu=t$, $\nu = z$ contribute:

\begin{align*}
\left(\dv[2]{n^{z}}{\tau}\right)
&= - R^{z}{}_{tzt}\, u^{t}u^{t} n^{z} \\
&= - R^{z}{}_{tzt}\, (1)(1)\, r_{\ast} \\
&= - (R^{+}_{z+z+})_{\text{obs}}\; r_{\ast} \\
&= - \frac{2GM}{r^{3}}\, r_{\ast} \, .
\end{align*}

So
\[
\boxed{
\left(\dv[2]{n^{z}}{\tau^{2}}\right)
 = -\frac{2GM}{r^{3}}\, r_{\ast}
}
\]
which is the same as in the margin note
``(same as Newtonian result)'' on the professor's sheet.

\subsubsection*{(b) Disruption radius}

Let the magnitude of the tidal acceleration at the star's surface equal
the surface gravity of the star, denoted here by $g_{\ast}$:

\[
\left|\dv[2]{n^{z}}{\tau^{2}}\right|_{\text{disrupt}}
 = g_{\ast}
 = \frac{GM_{\ast}}{r_{\ast}^{2}} \, ,
\]
where $M_{\ast}$ is the mass of the star.

From part (a),
\[
\left|\dv[2]{n^{z}}{\tau^{2}}\right|
 = \frac{2GM}{r^{3}}\, r_{\ast} \, .
\]

Setting these equal and solving for $r$ (which the sheet calls
$r_{\text{disrupt}}$),

\begin{align*}
\frac{2GM}{r_{\text{disrupt}}^{3}}\, r_{\ast}
 &= \frac{GM_{\ast}}{r_{\ast}^{2}} \\[4pt]
r_{\text{disrupt}}^{3}
 &= \frac{2M}{M_{\ast}}\, r_{\ast}^{3} \, .
\end{align*}

Therefore

\[
\boxed{
r_{\text{disrupt}}
 = \left( \frac{2M}{M_{\ast}} \right)^{1/3} r_{\ast}
}
\]

This is the expression the professor uses on the last lines of the
sheet to estimate the disruption radius for:
\begin{itemize}
\item the Sun falling into Sgr A* (with $M \sim 4\times 10^{6}M_{\odot}$),
\item the Sun falling into a $10M_{\odot}$ black hole.
\end{itemize}

\examnote{The structure to remember:
tidal acceleration $\sim (GM/r^{3})\times\text{(size)}$,
so $r_{\text{disrupt}} \sim (M/M_{\ast})^{1/3}r_{\ast}$.}

\end{solutionbox}

\end{document}
