\documentclass[10pt]{article}
\usepackage[margin=0.5in]{geometry}
\usepackage{amsmath,amssymb,bm,mathtools}
\usepackage{multicol}
\setlength{\columnsep}{12pt}
\setlength{\parindent}{0pt}
\setlength{\parskip}{2pt}
\setlength{\abovedisplayskip}{4pt}
\setlength{\belowdisplayskip}{4pt}
\setlength{\abovedisplayshortskip}{2pt}
\setlength{\belowdisplayshortskip}{2pt}
\raggedcolumns
% compact heading macro with a bit more top space
\newcommand{\chead}[1]{\vspace{6pt}{\large \textbf{#1}}\par}

\begin{document}
\small
% tighten math display spacing (re-assert after font size change)
\setlength{\abovedisplayskip}{3pt}
\setlength{\belowdisplayskip}{3pt}
\setlength{\abovedisplayshortskip}{2pt}
\setlength{\belowdisplayshortskip}{2pt}
\begin{multicols*}{2}

% =========================
\chead{Ch 1-2.}

For $c=1$, we have
\[
\beta=\frac{v}{c}=v
\qquad\Rightarrow\qquad
\gamma=\frac{dt}{d\tau}=\frac{1}{\sqrt{1-v^2}}
\]

\textit{S.R. Time Dilation:} 
\[
d\tau=dt\sqrt{1-v^2}\quad\text{ where } v^2=v_x^2+v_y^2+v_z^2
\]


\textit{Lorentz transformation matrix:}
\[
\Lambda_\nu^\mu=\left[\begin{array}{cccc}\gamma & -\gamma \beta & 0 & 0 \\ -\gamma \beta & \gamma & 0 & 0 \\ 0 & 0 & 1 & 0 \\ 0 & 0 & 0 & 1\end{array}\right]\Leftrightarrow
(\Lambda^{-1})_\nu^\mu=\left[\begin{array}{cccc}\gamma & \gamma \beta & 0 & 0 \\ \gamma \beta & \gamma & 0 & 0 \\ 0 & 0 & 1 & 0 \\ 0 & 0 & 0 & 1\end{array}\right]
\]


% =========================
\chead{Ch 3. Four Vector}

\textbf{3.1 Four Vector Notation}

\textit{Proper time:} $\tau$

\textit{Four-position:} $x^\mu(\tau)=(t(\tau),\,x(\tau),\,y(\tau),\,z(\tau))$

\textit{Four-displacement (arc length $s$):}
\[
d\mathbf{s}=\begin{bmatrix}dt\\ dx\\ dy\\ dz\end{bmatrix}
\]

\textit{Four-velocity:}
\[
\mathbf{u}\equiv
\begin{bmatrix}u^t\\ u^x\\ u^y\\ u^z\end{bmatrix}
\equiv
\begin{bmatrix}\dfrac{dt}{d\tau}\\[4pt]\dfrac{dx}{d\tau}\\[4pt]\dfrac{dy}{d\tau}\\[4pt]\dfrac{dz}{d\tau}\end{bmatrix}
=
\begin{bmatrix}\dfrac{dt}{dt\sqrt{1-v^2}}\\[4pt]
\dfrac{dx}{dt\sqrt{1-v^2}}\\[4pt]
\dfrac{dy}{dt\sqrt{1-v^2}}\\[4pt]
\dfrac{dz}{dt\sqrt{1-v^2}}
\end{bmatrix}
=
\begin{bmatrix}\gamma\\ v_x\gamma\\ v_y\gamma\\ v_z\gamma\end{bmatrix}.
\]

\textbf{3.2 Lorentz Transformation}
\[
u'^{\mu}=
\begin{bmatrix}
u'^t\\ u'^x\\ u'^y\\ u'^z
\end{bmatrix}
=
\begin{bmatrix}
\gamma & -\gamma\beta & 0 & 0\\
-\gamma\beta & \gamma & 0 & 0\\
0 & 0 & 1 & 0\\
0 & 0 & 0 & 1
\end{bmatrix}
\begin{bmatrix}
u^t\\ u^x\\ u^y\\ u^z
\end{bmatrix},
\]
\textbf{3.3 Scalar product, magnitude, the interval}

\textit{Minkowski metric:} $\eta_{\mu\nu}=\mathrm{diag}(-,1,1,1)$
\[
\begin{aligned}
\mathbf{A}\cdot\mathbf{B}&=\eta_{\mu\nu}A^\mu B^\nu  \quad\text{ or }\quad A_\mu B^\mu\\
&=\eta_{tt}A^tB^t+\eta_{xx}A^xB^x +\eta_{yy}A^yB^y+\eta_{zz}A^zB^z\\
&=-A^tB^t+A^xB^x+A^yB^y+A^zB^z\\
A^2=\mathbf{A}\cdot\mathbf{A}
&=-(A^t)^2+(A^x)^2+(A^y)^2+(A^z)^2.
\end{aligned}
\]

\textit{Interval}
\[
ds^2=\eta_{\mu\nu}\,dx^\mu dx^\nu
= -dt^2+dx^2+dy^2+dz^2.
\]

\textit{Normalization of 4-velocity}
\[
\begin{aligned}
ds^2&=-d\tau^2\ \ (\text{rest frame})\\
\Rightarrow\quad
\mathbf{u}\cdot\mathbf{u}
&=\eta_{\mu\nu}\frac{dx^\mu}{d\tau}\frac{dx^\nu}{d\tau}\\
&=\frac{1}{d\tau^2}\eta_{\mu\nu}dx^\mu dx^\nu\\
&=\frac{ds^2}{d\tau^2}=\boxed{-1}.
\end{aligned}
\]

\textbf{3.4 Relation Between $\mathbf{u}$ and $\mathbf{v}$}


Velocity relation: $v_i=\dfrac{u^i}{u^t}$ for $i=x,y,z$

At rest: 
\[
u^\mu=(1,0,0,0)
\]

Non-relativistic limit ($v\ll1$, $\gamma\approx1$): 
\[
u^\mu\approx(1,\,v_x,\,v_y,\,v_z)
\]

\textbf{3.5 Four Momentum}

\textit{4-Momentum for mass $m$ and light} 

\[ 
\mathbf{p}=m\mathbf{u} \quad\Rightarrow\quad p^\mu =
\begin{bmatrix}mu^t\\ mu^x\\ mu^y\\ mu^z\end{bmatrix} =
\begin{bmatrix}\gamma m\\ \gamma m v_x\\ \gamma m v_y\\ \gamma m v_z\end{bmatrix} =
\underbrace{\begin{bmatrix}E\\ Ev_x\\ Ev_y\\ Ev_z\end{bmatrix}}_{\text{light}}
\]

\textit{Relativistic energy:} \boxed{E=p^t=\gamma m}

\textit{Invariant mass relation:}
\[
\begin{aligned}
    \mathbf{p}\cdot\mathbf{p}
    &=-(p^t)^2+\vec{p}^{\,2}
    =-E^2+\vec{p}^{\,2}\quad\text{(1)}\\
    \mathbf{p}\cdot\mathbf{p}
    &=\eta_{\mu\nu}p^\mu p^\nu
    =m^2\eta_{\mu\nu}u^\mu u^\nu
    =-m^2\quad\text{(2)}
\end{aligned}
\]

where $\vec{p}=(p^x,p^y,p^z)$ for spatial 3-momentum.
\[
(1)=(2)\quad\Longrightarrow\quad\boxed{E^2-\vec{p}^{\,2}=m^2}
\]

\textbf{3.6 Energy by observer}

\textit{In rest frame (IRF):} Let $u^\mu_{\text{obs}}=(1,0,0,0)$, for passing object with 4-momentum $p^\mu$:
\[
\begin{aligned}
E_{(\text{obs})}&=-\,\mathbf{p}\cdot\mathbf{u}_{\text{obs}}\\
&=-\,\eta_{\mu\nu}p^\mu u^\nu_{\text{obs}}= p^t.
\end{aligned}
\]

\textit{Kinetic energy:} $E=m+\mathrm{KE}$ for $v\ll1$

% =========================
\chead{Ch 4. Index Notation}

\textbf{4.1 Useful Identity}
\begin{itemize}
\item identity matrix: $\delta^\mu_{\ \nu}=\mathrm{diag}(1,1,1,1)$
\item inverse transformation: $(\Lambda^{-1})^\mu_{\ \alpha}\,\Lambda^\alpha_{\ \nu}=\delta^\mu_{\ \nu}$
\item selector action: $\delta^\mu_{\ \nu}A^\nu=A^\mu$
\item derivative rule:
\[
\begin{aligned}
\frac{d}{d\tau}(A^2)
&=2\,\eta_{\mu\nu}A^\mu\frac{dA^\nu}{d\tau}.
\end{aligned}
\]
\end{itemize}

% =========================
\chead{Ch 5. Arbitrary Coordinates}

\textbf{5.1 Coordinate basis}

In any arbitrary curvilinear coordinates ($d s^2\neq d x^2+d y^2$), we define a coordinate basis $\mathbf e_\mu$ such that
\[
ds=dx^{\mu}\,\mathbf e_{\mu}
=\underbrace{du\,\mathbf e_u+dw\,\mathbf e_w}_{\text{2D case}}
\]


\textit{Metric tensor:}
\[
ds^2=g_{\alpha\beta}\,dx^{\alpha}dx^{\beta}
\]
where $g_{\alpha \beta} \equiv \mathbf{e}_\alpha \cdot \mathbf{e}_\beta$ comprises the metric tensor.

\textbf{5.2 Transformations}
Consider $u, w$ and new coordinates $u'(u, w)$ and $w'(u, w)$
\[
dx'^\mu=\frac{\partial x'^\mu}{\partial x^{\nu}}\,dx^{\nu},\quad\\
\]
For any vector $\mathbf{A}$:
\[
\boxed{A'^\mu=\frac{\partial x'^\mu}{\partial x^{\nu}}A^{\nu}\quad\Longleftrightarrow\quad A^\mu=\frac{\partial x^\mu}{\partial x^{\prime \nu}} A^{\prime \nu}}
\]
\textbf{5.3 Coordinate Transformations in Flat Spacetime}

For \textit{flat spacetime}, the general transformation law = L.T.:
\[
\begin{aligned}  \frac{\partial x^{\prime \mu}}{\partial x^\nu}=\Lambda_\nu^\mu \quad\Longleftrightarrow\quad \frac{\partial x^\mu}{\partial x^{\prime \nu}}=(\Lambda^{-1})_\nu^\mu\end{aligned}
\]

\textbf{5.4 The Metric for a Spherical Surface.}
In polar coordinates, the metric on a sphere of radius $R$ is:
\[
ds^2=R^2\,d\theta^2 + R^2\sin^2\!\theta\,d\phi^2
\]
that is
\[g_{\mu \nu}=\left[\begin{array}{ll}g_{\theta \theta} & g_{\theta \phi} \\ g_{\phi \theta} & g_{\phi \phi}\end{array}\right]=\left[\begin{array}{cc}R^2 & 0 \\ 0 & R^2 \sin ^2 \theta\end{array}\right]\]

% =========================
\chead{Ch 6. Tensor Equations}

\textbf{6.1 Vectors vs covectors}

\textit{Vector transformation law:}
\[
A'^\mu=\frac{\partial x'^\mu}{\partial x^{\nu}}A^{\nu}
\]

\textit{Covector transformation law:}

$A_\mu$ transforms as a covector:
$$
\begin{aligned}
A_\mu^{\prime} & =g_{\mu \nu}^{\prime} A^{\prime \nu}=\left(\frac{\partial x^\alpha}{\partial x^{\prime \mu}} \frac{\partial x^\beta}{\partial x^{\prime \nu}} g_{\alpha \beta}\right)\left(\frac{\partial x^{\prime \nu}}{\partial x^\gamma} A^\gamma\right) \\
& =\frac{\partial x^\alpha}{\partial x^{\prime \mu}} \underbrace{\left(\frac{\partial x^\beta}{\partial x^{\prime \nu}} \frac{\partial x^{\prime \nu}}{\partial x^\gamma}\right)}_{\delta^\beta{ }_\gamma} g_{\alpha \beta} A^\gamma=\frac{\partial x^\alpha}{\partial x^{\prime \mu}} g_{\alpha \gamma} A^\gamma \\
& =\frac{\partial x^\alpha}{\partial x^{\prime \mu}} A_\alpha
\end{aligned}
$$

\textbf{6.2 Gradient and lowering}

\textit{Gradient of a scalar is a covector:}
\[
\partial_\mu\phi=\frac{\partial\phi}{\partial x^{\mu}}\ (\text{covector})
\]

where
$\partial_\mu \equiv \frac{\partial}{\partial x^\mu} = \left(\frac{\partial}{\partial t}, \frac{\partial}{\partial x}, \frac{\partial}{\partial y}, \frac{\partial}{\partial z}\right)_\mu$

\textit{Lowering an index with the metric produces a covector:}
$$
\begin{aligned}
A_\mu=g_{\mu\nu}A^{\nu}
\end{aligned}
$$
\textbf{6.3 Scalar contraction (invariant)}

\textit{Scalar contraction is invariant:}
\[A^{\prime \mu} B_\mu^{\prime}=\left(\frac{\partial x^{\prime \mu}}{\partial x^\alpha} A^\alpha\right)\left(\frac{\partial x^\beta}{\partial x^{\prime \mu}} B_\beta\right)=\underbrace{\left(\frac{\partial x^{\prime \mu}}{\partial x^\alpha} \frac{\partial x^\beta}{\partial x^{\prime \mu}}\right)}_{\delta^\beta{ }_\alpha} A^\alpha B_\beta=A^\alpha B_\alpha\]

\textbf{6.4 Inverse metric}

For any metric tensor $g_{\mu \nu}$:
\[
\boxed{g_{\mu \nu}^{\prime}=\frac{\partial x^\alpha}{\partial x^{\prime \mu}} \frac{\partial x^\beta}{\partial x^{\prime \nu}} g_{\alpha \beta} \quad\Longleftrightarrow\quad g_{\alpha \beta}=\frac{\partial x^{\prime \mu}}{\partial x^\alpha} \frac{\partial x^{\prime \nu}}{\partial x^\beta} g_{\mu \nu}^{\prime}}
\]
and also the inverse:
\[
\boxed{g'^{\mu\nu}=\frac{\partial x'^\mu}{\partial x^{\alpha}}\frac{\partial x'^\nu}{\partial x^{\beta}}\,g^{\alpha\beta} \quad\Longleftrightarrow\quad g^{\alpha\beta}=\frac{\partial x^{\prime \alpha}}{\partial x^\mu} \frac{\partial x^{\prime \beta}}{\partial x^\nu} g'^{\mu\nu}}
\]

\textbf{6.5 Tensor (master law)}
\[
T'^{\mu\nu\dots}{}_{\alpha\beta\dots}
=\frac{\partial x'^\mu}{\partial x^{r}}\frac{\partial x'^\nu}{\partial x^{\sigma}}\cdots
\frac{\partial x^{\gamma}}{\partial x'^\alpha}\frac{\partial x^{\delta}}{\partial x'^\beta}\cdots
\,T^{r\sigma\dots}{}_{\gamma\delta\dots}.
\]

% =========================
\chead{Ch 7. Maxwell}
\textbf{7.1 Charge density and current density}

\textit{Length contracts:}
$$V=V^{\prime} / \gamma \quad\Longrightarrow\quad \rho=\gamma \rho^{\prime}$$
\textit{Four-current density:}
$$\boxed{J^\mu=\left(\rho, \rho v_x, \rho v_y, \rho v_z\right) \quad\text{ with }\vec{J}=\rho \vec{v}}$$

\textbf{7.2 Gauss's law}

(differential form):
\[
\nabla \cdot \mathbf{E}=\frac{\partial E_x}{\partial x}+\frac{\partial E_y}{\partial y}+\frac{\partial E_z}{\partial z}=4 \pi k \rho
\]
where E is the electric field:
\[
\frac{d \mathbf{p}}{d t}=\mathbf{F}_e=q \mathbf{E} \quad\Longrightarrow\quad 
\boxed{\frac{d p^\mu}{d \tau}=q F^\mu{ }_\nu u^\nu}
\]



\textbf{7.3 Maxwell equations}
\[
\boxed{\ \partial_\nu F^{\mu\nu}=4\pi k\,J^\mu\ },\qquad
\partial_\sigma F_{\mu\nu}+\partial_\mu F_{\nu\sigma}+\partial_\nu F_{\sigma\mu}=0.
\]
where F is the field tensor:
\[
F_{\mu\nu}=\begin{bmatrix}
0 & E_x & E_y & E_z\\
-E_x & 0 & B_z & -B_y\\
-E_y & -B_z & 0 & B_x\\
-E_z & B_y & -B_x & 0
\end{bmatrix}
\]

\textit{Charge conservation:} Take $\partial_\mu$ of both sides and commute derivatives: LHS = 0 because $F$ is antisymmetric,
$$
\partial_\mu \partial_\nu F^{\mu \nu} \equiv 0\quad\Longrightarrow\quad \boxed{\partial_\mu J^\mu=0}
$$

\textbf{7.4 Potentials}
\[
F_{\mu\nu}=\partial_\mu A_\nu-\partial_\nu A_\mu.
\]

% =========================
\chead{Ch 8. Geodesics}

\textbf{8.0 Lagrange equation:}
Lagrangian for our worldline is:
\[
L(x, \dot{x}) \equiv \sqrt{-g_{\mu \nu}(x) \dot{x}^\mu \dot{x}^\nu}, \quad \dot{x}^\mu \equiv \frac{d x^\mu}{d \sigma}
\]
Euler-Lagrange equation for our worldline is:
\[
\frac{d}{d\sigma}\left(\frac{\partial L}{\partial \dot{x}^\mu}\right)-\frac{\partial L}{\partial x^\mu}=0
\]

\textbf{8.1 Extremal proper time}
Between two a timelike geodesic two timelike-separated events $A \to B$, the \textit{proper time} is:
\[
\tau=\int\!\sqrt{-g_{\mu\nu}\,\frac{dx^{\mu}}{d\sigma}\frac{dx^{\nu}}{d\sigma}}\,d\sigma.
\]

\textbf{8.2 Geodesic equations}
\[
\boxed{\begin{aligned}
0&=\frac{d}{d\tau}\!\Big(g_{\alpha\nu}\,\frac{dx^{\nu}}{d\tau}\Big)
-\frac{1}{2}\,\partial_{\alpha}g_{\mu\nu}\,\frac{dx^{\mu}}{d\tau}\frac{dx^{\nu}}{d\tau}\\[2pt]
&=\frac{d^2 x^\rho}{d \tau^2}+\underbrace{\frac{1}{2} g^{\rho \alpha}\left(\partial_\mu g_{\alpha \nu}+\partial_\nu g_{\alpha \mu}-\partial_\alpha g_{\mu \nu}\right)}_{\Gamma^\rho{ }_{\mu \nu}} \frac{d x^\mu}{d \tau} \frac{d x^\nu}{d \tau}
\end{aligned}}
\]

\textbf{8.3 Normalization}
\[
u\!\cdot\!u=g_{\mu\nu}\,\frac{dx^{\mu}}{d\tau}\frac{dx^{\nu}}{d\tau}=-1\ (\text{timelike}),\quad =0\ (\text{null}).
\]

% =========================
\chead{Ch 9. Schwarzschild Metric}
\textbf{9.0 Spherical Coordinates for Flat Spacetime:}
\[
ds^2= -dt^2 + dr^2 + r^2 d\theta^2 + r^2\sin^2\!\theta\,d\phi^2
\]


\textbf{9.1 Schwarzschild Metric}
\[
\boxed{\ ds^2 = \underbrace{-\Big(1-\frac{2GM}{r}\Big)}_{g_{t t} \text{ time part}} dt^2 + \underbrace{\Big(1-\frac{2GM}{r}\Big)^{-1}}_{g_{r r}\text{ radial part}}\!dr^2 + \underbrace{r^2}_{g_{\theta \theta}} d\theta^2 + \underbrace{r^2\sin^2\!\theta\,d\phi^2}_{g_{\phi \phi}}\ }.
\]

\textit{Features:} spherically symmetric, static, vacuum, and becomes the flat space metric in the limit as $r \rightarrow \infty$.

\textit{Units:} $G$ has units of $\mathrm{m} / \mathrm{kg}$, and $GM$ in units of $\mathrm{m}$.

\textbf{9.2 Meaning of $r$}
\[
r=\frac{C}{2\pi}\ (\text{circumference}),\qquad ds=\frac{dr}{\sqrt{1-2GM/r}}\ (t=\text{const}).
\]

\textbf{9.3 Newtonian limit and $r_s$}
\[
\frac{d^2 r}{d\tau^2}\Big|_{\text{rest}}= -\,\frac{1}{2}\,\frac{r_s}{r^2}\ \Rightarrow\ r_s=2GM.
\]

\textbf{9.4 Gravitational time dilation \& redshift}

\textit{Time dilation} in Schwarzschild metric:
\[
\boxed{\Delta\tau_r=\sqrt{1-\tfrac{2GM}{r}}\,\Delta t}
\]
\textit{Redshift} in Schwarzschild metric:
\[
\begin{aligned}
\frac{\lambda_R}{\lambda_E}&=\sqrt{\frac{1-\tfrac{2GM}{r_R}}{1-\tfrac{2GM}{r_E}}}\approx 1+g\,h 
\end{aligned}
\]

if $\frac{2 G M}{r} \ll 1$ and $h \equiv r_R-r_E \ll r_E$. Here $g=GM/r_E^2$.

% =========================
\chead{Ch 10. Particle Orbits}

\textbf{10.1 Conserved quantities (equatorial)}
\[
e=\Big(1-\frac{2GM}{r}\Big)\frac{dt}{d\tau},\qquad l=r^2\frac{d\phi}{d\tau}.
\]

\textbf{10.2 Radial energy form}
\[
\tfrac12\Big(\frac{dr}{d\tau}\Big)^2=E-\Big[\,\frac{GM}{r}+\frac{l^2}{2r^2}-\frac{GMl^2}{r^3}\,\Big],\quad E=\tfrac12(e^2-1).
\]

\textbf{10.3 Circular orbits \& Kepler}
\[
\frac{dV}{dr}=0\ \Rightarrow\ r_c=\frac{l^2}{2GM}\Big(1\pm\sqrt{1-12(GM/l)^2}\Big),\quad
\Omega^2=\frac{GM}{r_c^3}.
\]

\textbf{10.4 Acceleration and ISCO}
\[
\frac{d^2 r}{d\tau^2}= -\,\frac{GM}{r^2}+\frac{l^2}{r^3}-\frac{3GMl^2}{r^4},\quad r_{\text{ISCO}}=6GM,\ e_{\text{ISCO}}=\sqrt{8/9}.
\]

% =========================
\chead{Ch 12. Photon Orbits}

\textbf{12.1 Impact parameter}
\[
\boxed{\ b=\frac{l}{e}=
r^2\left(1-\frac{2 G M}{r}\right)^{-1} \frac{d \phi}{d t} }
\]

or $$
\frac{d \phi}{d t}=\frac{1-2 G M / r}{r^2} b
$$

\textbf{12.2 Equatorial plane}
$$
\begin{aligned}
&\theta=\frac{\pi}{2}, \quad d \theta=0, \quad \sin \theta=1
\end{aligned}
$$

\textbf{12.3 Equation of Radial Motion for a Photon}
Use the Schwarzschild line element on the equatorial plane ($\theta=\pi/2$) and the null condition $ds^2=0$:
$$
\begin{aligned}
0 & =-\left(1-\frac{2 G M}{r}\right) d t^2+\left(1-\frac{2 G M}{r}\right)^{-1} d r^2+r^2 d \phi^2 \\
& \Rightarrow 1=\frac{1}{\left(1-\frac{2 G M}{r}\right)^2}\left(\frac{d r}{d t}\right)^2+\left(1-\frac{2 G M}{r}\right) \frac{b^2}{r^2}
\end{aligned}
$$

Divide by $b^2$ to obtain the \textit{energy‑like} form with an effective potential $V(r)$:
$$\boxed{\ \frac{1}{b^2}\;=\;\underbrace{\frac{1}{b^2\,\big(1-\tfrac{2GM}{r}\big)^{2}}\Big(\frac{dr}{dt}\Big)^2}_{\text{radial “kinetic” term}}\; +\; \underbrace{\frac{1-\tfrac{2GM}{r}}{r^2}}_{\displaystyle V(r)}\ }$$
\textit{Remark:}
\begin{itemize}
\item The potential $V(r)=(1-2GM/r)/r^2$ has a peak at $r=3GM$
\item Photons with $1/b^2$ larger than this peak spiral inward, with the \textit{critical case} $b^2=27(GM)^2$ corresponding to the unstable circular photon orbit at $r=3GM$.
\end{itemize}

Setting $GM\to0$ gives flat space result:
$$\boxed{\ \frac{1}{b^2}=\frac{1}{b^2}\Big(\frac{dr}{dt}\Big)^2 + \frac{1}{r^2}\quad\text{(flat space)} }$$
which says a light ray from infinity always returns to infinity

\textbf{12.4 Locally orthonormal frames (LOFs) for static observers}

A stationary observer constructs a \textit{locally orthonormal tetrad} $\{\mathcal O_t,\mathcal O_x,\mathcal O_y,\mathcal O_z\}$ at a point $P$, satisfying
$$\boxed{\ \mathcal O_x\!\cdot\!\mathcal O_x = \mathcal O_y\!\cdot\!\mathcal O_y = \mathcal O_z\!\cdot\!\mathcal O_z = +1,\quad \mathcal O_t\!\cdot\!\mathcal O_t=-1,\quad \mathcal O_a\!\cdot\!\mathcal O_b=0\ (a\neq b) }$$
so the metric in the LOF is Minkowski:
$$\boxed{\ (g_{\mu\nu})_{\text{LOF}} = \eta_{\mu\nu} }$$
If $A$ is any four‑vector with components $A^\mu$ in Schwarzschild coordinates and $(\mathcal O_a)^\mu$ are the **Schwarzschild components** of the LOF basis vectors, the observed components follow from scalar products:
$$\boxed{\ A_{\text{obs}}^{\,t}= -\,\mathcal O_t\!\cdot\!A,\quad A_{\text{obs}}^{\,x}= \mathcal O_x\!\cdot\!A,\quad A_{\text{obs}}^{\,y}= \mathcal O_y\!\cdot\!A,\quad A_{\text{obs}}^{\,z}= \mathcal O_z\!\cdot\!A }$$
For a static observer aligned with the Schwarzschild directions $(\phi,-\theta,r)$, the tetrad components are
$$\begin{aligned}
(\mathcal O_t)^\mu&=\Big(\frac{1}{\sqrt{1-\tfrac{2GM}{r}}},0,0,0\Big),\\
(\mathcal O_x)^\mu&=\Big(0,\frac{1}{r\sin\theta},0,0\Big),\\
(\mathcal O_y)^\mu&=\Big(0,0,\frac{1}{r},0\Big),\\
(\mathcal O_z)^\mu&=\Big(0,0,0,\sqrt{1-\tfrac{2GM}{r}}\Big)
\end{aligned}
$$
These satisfy $\mathcal O_a\!\cdot\!\mathcal O_b=\eta_{ab}$ with the Schwarzschild metric.


\textbf{12.4 Photon 4-momentum (equatorial)}

\textit{Photon 4-momentum} in the equatorial plane:
$$
p^\mu=\frac{E}{1-2 G M / r} \frac{d x^\mu}{d t} \quad\text{where } E\equiv m e.
$$

\textit{Components:}
$$
\begin{aligned}
p^t&=\frac{E}{(1-2 G M / r)}\\
p^\phi &=\frac{E}{(1-2 G M / r)} \frac{d \phi}{d t}=E \frac{b}{r^2} \\
p^\theta &=0\\
p^r&=\frac{E}{(1-2 G M / r)} \frac{d r}{d t}= \pm E \sqrt{1-\frac{b^2}{r^2}\left(1-\frac{2 G M}{r}\right)}
\end{aligned}
$$
where ($+$) for outgoing and ($-$) for ingoing photons.



\textbf{12.5 Critical escape angle for stationary observers}

Let a static observer at radius $r$ emit a photon in the equatorial plane making an angle $\psi$ with the \textit{radially outward} direction ($\mathcal O_z$). In the LOF, the photon speed is 1, so
$$\boxed{\ \sin\psi=\frac{v_{x,\text{obs}}}{1}=\frac{(\mathcal O_x\!\cdot\!p)}{-\,(\mathcal O_t\!\cdot\!p)} }$$
Using (12.10) and (12.12) in the dot products and imposing the \textit{critical} condition $b^2=27(GM)^2$ (the top of $V(r)$) gives
$$\boxed{\ \sin\psi_c\;=\;\frac{3\sqrt{3}\,GM}{r}\,\sqrt{1-\frac{2GM}{r}}\ }$$

\begin{itemize}
    \item As $r\to\infty$, $\sin\psi_c\to0$ so $\psi_c\to180^\circ$ (escape in almost any direction). 
    \item At $r=3GM$, $\sin\psi_c=1\Rightarrow\psi_c=90^\circ$ (photon sphere).
    \item As $r\to2GM$, $\psi_c\to0^\circ$: even radially outward light is captured.
\end{itemize}

\textbf{12.6 Energy measured by a stationary observer (blueshift)}

The energy measured in the LOF is $E_{\text{obs}}\equiv -\,\mathcal O_t\!\cdot\!p$. Using (12.10) and (12.12),
$$\boxed{\ E_{\text{obs}}\;=\;\frac{E}{\sqrt{1-\tfrac{2GM}{r}}}\;>\;E}$$
which diverges as $r\to2GM$. 
This blueshift reflects that \textit{local clocks run slower} than the clock at infinity. Conversely, light sent upward is \textit{redshifted} as seen at infinity.


% =========================
\chead{Ch 14. Event Horizon}

\textbf{14.1 Horizon signals}

\textit{Stationary worldline} at $r=2GM$ (lightlike):

Set $dr=d\theta=d\phi=0$: 
\[
\begin{aligned}
ds^2\Big|_{\text{static}}= -\Big(1-\frac{2GM}{r}\Big)dt^2\Rightarrow ds^2=0
\end{aligned}
\]

\textit{Infinite redshift:}
$$\frac{\lambda_R}{\lambda_E}=\sqrt{\frac{1-\tfrac{2GM}{r_R}}{1-\tfrac{2GM}{r_E}}}\to\infty\  \ (r_E\to2GM^+)$$

\textbf{14.2 Coordinate freeze at the horizon}
\[
\frac{dr}{dt},\ \frac{d\phi}{dt}\ \propto\ (1-2GM/r)\ \to\ 0\quad (r\to2GM^+).
\]

\textbf{14.3 Finite distances/times}

\textit{Finite radial proper distance to the horizon:}
\[
\begin{aligned}
\mathcal D( R\to 2GM)&=\int_{2GM}^{R}\!\frac{dr}{\sqrt{1-2GM/r}}\\[2pt]
&=\sqrt{R(R-2GM)}+GM\ln\!\Bigg|\frac{\sqrt{R}+\sqrt{R-2GM}}{\sqrt{2GM}}\Bigg|.
\end{aligned}
\]


\textbf{14.4 Finite proper time to fall past the horizon}

For a particle released from rest at $r=R$, the proper time to $r=0$ (passing $2GM$ en route) is
\[
\Delta\tau\Big|_{\text{rest at }R\to0}=\frac{\pi R^{3/2}}{\sqrt{8GM}}
\]


For a solar‑mass black hole and $R=1000\,GM$, this gives $\Delta\tau\approx0.18\,\mathrm{s}$.

\textbf{14.5 The future inside the horizon is $r=0$}

Use the radial acceleration equation from Ch. 10 (valid for all $r$):
$$\boxed{\ \frac{d^2 r}{d\tau^2}= -\,\frac{GM}{r^2}+\frac{l^2}{r^3}-\frac{3GMl^2}{r^4}\ }$$
For $r<3GM$ the bracketed combination is negative, so $d^2r/d\tau^2<0$ irrespective of $l$ (any freely‑falling worldline inevitably moves inward).

\textit{Maximum proper time:}
$$\Delta\tau_{\max}(2GM\to0)=\pi GM \quad\text{(inside the horizon).}$$
thus the future of any infaller inside the horizon ends at the singularity $r=0$ after a finite proper time.

\end{multicols*}
\end{document}
