%========================================================
% Grad School CV (Applied Math / Physics / ML Hybrid)
% Author: <Your Name>
% Version: 2025-07-17
%========================================================
\documentclass[11pt]{article}

%---------- Encoding & Fonts ----------

\usepackage[utf8]{inputenc}
\usepackage[T1]{fontenc}
\usepackage{mathpazo}      % Palatino text + math
\usepackage{microtype}     % Better kerning & spacing



%---------- Page Geometry ----------
\usepackage[margin=0.8in,includefoot]{geometry}
\setlength{\parindent}{0pt}
\setlength{\parskip}{4pt}

%---------- Packages ----------
\usepackage{enumitem}
\usepackage{hyperref}
\usepackage{array}
\usepackage{multirow}
\usepackage{graphicx}
\usepackage{xcolor}

% Hyperlink colors (subtle)
\definecolor{linkblue}{HTML}{005F9E}
\hypersetup{
  colorlinks=true,
  urlcolor=linkblue,
  linkcolor=linkblue,
  citecolor=linkblue
}

%---------- Section Formatting ----------
\newcommand{\cvsection}[1]{%
  \vspace{1.25em}%
  \textbf{\large #1}%
  \vspace{0.5em}\hrule\vspace{0.6em}
}

%---------- Entry Macros ----------
% Single-line heading: Role | Place | Dates
\newcommand{\cvheading}[3]{\textbf{#1} \hfill #2 \hfill #3}

% Multi-field entry (degree etc.)
\newcommand{\cvedu}[5]{%
  \textbf{#1}, #2 \hfill #3\\%
  #4 \\
  #5 \par
}

% Bullet list with tight spacing
\newenvironment{cvitems}{\begin{itemize}[leftmargin=*,topsep=2pt,itemsep=2pt]}{\end{itemize}}

% Inline tag (for keywords)
\newcommand{\tag}[1]{\fcolorbox{linkblue!60}{linkblue!10}{\footnotesize\textsf{#1}}}

%---------- Document ----------
\begin{document}

%================ Header ================
{\LARGE \textbf{Erise He}} \hfill \today\\
Applied Mathematics \& Physics (BS candidate), Emory University (Expected May 2026)\\
Email: \href{mailto:you@emory.edu}{you@emory.edu} \quad|\quad Website: \href{https://your-site.example}{your-site.example} \quad|\quad GitHub: \href{https://github.com/yourhandle}{yourhandle} \quad|\quad Phone: (xxx) xxx-xxxx

%================ Research Interests ================
\cvsection{Research Interests}
Geometric Deep Learning; Symplectic / Differential Geometric Methods in ML; Topological Data Analysis (Persistent Homology) for Representation Dynamics; Neuro‑/Psychoanalytic Inspired AI Architectures (symbolic misrecognition, intrinsic “desire” dynamics); Optimal Transport for Representation Alignment.

%================ Education ================
\cvsection{Education}ß
\cvedu{Emory University}{Atlanta, GA}{BS Applied Mathematics; BS Physics (Double Major), GPA: 4.00/4.00}{Minor / Concentration: Philosophy (unofficial focus in Psychoanalysis)}{Expected May 2026}

\begin{cvitems}
  \item \textbf{Honors / Distinctions:} Dean’s List (Fall 2024); Undergraduate Research Scholar Award (Emory College, 2025).
  \item \textbf{Selected Advanced Coursework:} PDE I (A); PDE in Action (A); Real Analysis II (B); Differential Geometry (in progress Fall 2025); Machine Learning (in progress Fall 2025); Mathematical Statistics I/II; Abstract Algebra; Numerical Linear Algebra; Topology (audit); Physics: Classical Mechanics II, Quantum I, Statistical Mechanics.
  \item \textbf{Planned Senior Thesis:} “Subjector‑1: Symplectic–Topological Dynamics of Desire in Hybrid Symbolic/Latent AI Systems.”
\end{cvitems}

%================ Research Experience ================
\cvsection{Research Experience}

\cvheading{Undergraduate Research Assistant, Applied Math}{Emory University}{Jan 2025 -- Present}
\begin{cvitems}
  \item Working with Prof. <Advisor Name> on streaming persistent homology for high‑velocity symbolic sequences; implementing landmark VR complexes in Python/C++.
  \item Proving stability bounds for loop persistence under quantised optimal transport noise; preliminary note in preparation.
  \item Integrating PH descriptors with low‑dimensional latent dynamics driven by Hamiltonian vector fields (prototype for \emph{Subjector‑1}).
\end{cvitems}

\cvheading{Independent Project: Subjector‑1 Psychoanalytic AI Prototype}{Self‑directed / Emory}{Aug 2024 -- Present}
\begin{cvitems}
  \item Designed a hybrid architecture coupling LLM embeddings (semantic manifold) to discrete signifier graphs; injects “desire” as residual misprojection energy.
  \item Formulated desire flow as a bounded Hamiltonian on an exact symplectic latent phase space; showed Liouville volume preservation in discrete integrator regime.
  \item Built streaming Vietoris–Rips PH to detect recurrent 1‑cycles (fantasy / symbolic loops) in generated language trajectories; early results on synthetic corpora.
  \item Draft manuscript available: \href{https://your-site.example/subjector1.pdf}{Subjector‑1 preprint}.
\end{cvitems}

\cvheading{Philosophy Honors Project: Structure of Misrecognition in Lacan}{Emory Philosophy Dept.}{Jan 2024 -- Dec 2024}
\begin{cvitems}
  \item Mapped Lacanian registers (Symbolic / Imaginary / Real) into categorical relations; explored functorial analogues to projection / residue in representation.
  \item Outcome informs formal design of AI “desire loops.”
\end{cvitems}

%================ Publications & Manuscripts ================
\cvsection{Publications \& Manuscripts}
% Use reverse chronological; list submitted/under review clearly.
\begin{cvitems}
  \item E. He. \emph{Subjector‑1: A Symplectic–Topological Non‑convergent Prototype for Simulating Artificial Desire.} Draft manuscript, July 2025. [preprint link].
  \item E. He, <Advisor>. “Streaming Persistent Homology for Symbolic Graph Walks.” In preparation for submission to \emph{SIAM Journal on Applied Algebra and Geometry}, 2026.
\end{cvitems}

%================ Posters & Talks ================
\cvsection{Talks \& Posters}
\begin{cvitems}
  \item “Desire as Residual: Toward a Topological Subject in AI.” Emory Undergraduate Research Symposium, Apr 2025.
  \item (Planned) NeurIPS 2025 Workshop on Geometry and Learning -- Poster submission pending.
\end{cvitems}

%================ Technical Skills ================
\cvsection{Technical Skills}
\begin{cvitems}
  \item \textbf{Programming:} Python (NumPy, PyTorch, PyTorch Geometric, JAX basics), C++17, Julia (basic), MATLAB.
  \item \textbf{Math / ML Libraries:} giotto‑tda, Gudhi, ripser.py; geomstats; POT (Python Optimal Transport); networkx; scikit‑learn.
  \item \textbf{ML Ops / Tools:} Jupyter, Weights\&Biases, Git, Docker (basic).
  \item \textbf{Typesetting / Writing:} \LaTeX{}, TikZ‑cd, Overleaf.
\end{cvitems}

%================ Selected Projects (Optional condensed) ================
\cvsection{Selected Projects}
\cvheading{Hamiltonian Flow Integrator for Learned Latent Manifolds}{Python / PyTorch}{2025}
\begin{cvitems}
  \item Implemented symplectic Euler and leapfrog integrators over learned encoder embeddings; benchmarked energy drift vs. vanilla gradient dynamics.
\end{cvitems}

\cvheading{Optimal Transport Codebook Quantization}{Python}{2025}
\begin{cvitems}
  \item Used entropic OT (Sinkhorn) to align LLM latent distributions with discrete symbolic vocabularies; measured distortion vs. k‑means baselines.
\end{cvitems}

%================ Teaching & Service ================
\cvsection{Teaching \& Academic Service}
\cvheading{Undergraduate TA, Differential Equations}{Emory Math Dept.}{Fall 2024}
\begin{cvitems}
  \item Led weekly problem sessions; authored supplementary notes on PDE numerical schemes.
\end{cvitems}

\cvheading{Writing Fellow (Philosophy)}{Emory Writing Center}{2023 -- 2024}
\begin{cvitems}
  \item Assisted students in formal argument structure; experience translating abstract theory for wider audiences.
\end{cvitems}

%================ Awards & Scholarships ================
\cvsection{Awards}
\begin{cvitems}
  \item Emory Undergraduate Research Scholar Award (2025)
  \item Philosophy Department Best Essay Prize (2024)
  \item Dean’s List (multiple terms)
\end{cvitems}

%================ Professional Development ================
\cvsection{Professional Development}
\begin{cvitems}
  \item Summer School: Geometric Deep Learning (online intensive), 2025.
  \item Workshop: Topological Data Analysis in ML (SIAM AG), 2025.
\end{cvitems}

%================ References ================
\cvsection{References}
% Tailor per application; list 3-4.
\begin{tabular}{@{}p{0.47\linewidth}p{0.47\linewidth}@{}}
\textbf{Prof. <Math Advisor>} & \textbf{Prof. <Philosophy Mentor>}\\
Dept. of Mathematics, Emory & Dept. of Philosophy, Emory \\
Email: advisor@emory.edu & Email: mentor@emory.edu \\[0.5em]
\textbf{Prof. <External Collaborator>} & \textbf{Prof. <Physics Instructor>}\\
<Institution> & Emory Physics Dept. \\
Email: collab@univ.edu & Email: physics@emory.edu
\end{tabular}

\end{document}
